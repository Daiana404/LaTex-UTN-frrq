\documentclass[11pt,a4paper]{article}
\usepackage[utf8]{inputenc}
\usepackage[T1]{fontenc}
\usepackage{amsmath}
\usepackage{amssymb}
\usepackage{amsfonts}
\usepackage{graphicx}
\usepackage[spanish]{babel}
\usepackage{booktabs, fourier, tabularx, wrapfig, multicol,multirow,caption, subcaption,tikz, fancyhdr, steinmetz, xcolor}
\usepackage{xfrac}
\usepackage{soul}
\usepackage[many]{tcolorbox}

\usepackage[left=2cm,right=2cm,top=2cm,bottom=2cm]{geometry}

\graphicspath{{../Figuras}}

%Configuración y comandos

%tcolorbox personalizado
\newtcolorbox{cajita}{colback=white!97!brown, colframe=brown!15!gray, breakable}

%Encabezado y pie de página
\fancyhead[R]{\textsc{Mecánica de Fluidos y Máquinas Fluidodinámicas}}
\fancyhead[L]{Hoja de fórmulas}
\fancyfoot[R]{ \vspace{.1cm} Página \thepage }
\fancyfoot[L]{\hrule \vspace{.1cm} Ingeniería Electromecánica}
\fancyfoot[C]{\vspace{.1cm}\includegraphics[width=.1\textwidth]{../../utncom}}

\newtcolorbox{mybox}[1]{title = \textsc{Unidad #1},colbacktitle=red!85!black,enhanced,attach boxed title to top center={yshift=-2mm}, colbacktitle=white, coltitle=gray!50!black, boxed title style={colframe=blue!30},colback=white!97!brown, colframe=blue!30}

%Comando para fasores :)
\newcommand{\fasor}[1]{\uppercase{\textit{\textbf{#1}}}}
%Comando título de cada unidad
\newcommand{\unidad}[2]{\begin{center}
		\fontsize{10}{10}\selectfont\color{gray!50!black}\scshape Unidad #1 \\
		\fontsize{14}{14}\selectfont \scshape #2
	\end{center} \vspace{-.6cm}}
%Comando pra subtitulos
\newcommand{\subtitulo}[1]{
	\textbf{#1} \\ \vspace{.1cm} {\color{gray} \hrule} \vspace{.2cm}
}
%Comando para variables y sus unidades
\newcommand{\variable}[2]{$#1$ $\left[#2\right]$}
%Comando para grados
\newcommand{\grado}{^\circ}
%Comando para el volumen desplazado
\newcommand{\vdes}{\st{\textsl{V}}}

\begin{document}
	\pagestyle{fancy}
	\section*{Nomenclatura}
	
		\begin{center}
			\begin{tabular}{r l r l}
			\variable{V}{m^3} & Volumen & \variable{W}{kgf} & Peso\\
			\variable{\mu}{Pa \cdot s} & Viscosidad absoluta & \variable{v}{m^2/s} & Viscosidad cinemática\\
			\variable{\sigma}{N/m} & Tensión superficial & $\overline{GM}$ & Altura metacéntrica\\
			$G$ & Centro de gravedad & $C$ & Centro de presión\\
			\variable{\rho}{kg/m^3} & Densidad & $\rho_{rel}$ & Densidad relativa\\
			\variable{\tau}{N/m^2} & Esfuerzo de corte & $g=9.8 \sfrac{m}{s^2}$ & Aceleración de la gravedad\\
			\variable{W}{kgf}& Peso & \variable{\gamma}{kgf/m^3}& Peso específico\\
			\variable{J}{m^4} & Segundo momento & \variable{\overline{J}}{m^4} & Segundo momento respecto a G\\
		\end{tabular}
		\end{center}
	\section*{Conversión de unidades}
		\begin{tabular}{r l r l}
			Presión & & & \\
			Temperatura & $K =\ \grado C + 273.15$ & $ \grado R\ =\ \grado F + 459.67$ & \\
		\end{tabular}
	\unidad{1}{Conceptos generales}
		\begin{cajita}
			\centering
			\begin{multicols}{2}
				\subtitulo{Presión \vspace{.085cm}}
				\begin{tabular}{l l}
					\multicolumn{2}{c}{$P_{absoluta} = P_{atmosférica} + P_{manométrica}$\vspace{.1cm}}\\
					$P_{man} (+)$ & Presión manométrica\vspace{.1cm} \\
					$P_{man} (-)$ & Vacío \\
				\end{tabular}
				\columnbreak
				
				\subtitulo{Densidad y peso específico}
				$\rho_{rel} = \dfrac{\rho}{\rho_{H_2O}}$\vspace{.1cm}\\
				$\gamma = \dfrac{W}{V} = \dfrac{mg}{V} = \rho g$\\
			\end{multicols}
			
			\begin{multicols}{2}
				\subtitulo{Viscosidad\vspace{.08cm}}
				\begin{tabular}{l l}
					$\tau = \mu \dfrac{du}{dy}$ & $v = \dfrac{\mu}{\delta}$\\
					Fluido newtoniano & $\mu = cte$ \\
					Fluido ideal & $\mu = 0$
				\end{tabular}
			
			\columnbreak
			
				\subtitulo{Tensión superficial}
				\begin{tabular}{l l}
					No sé que pingo poner acá & help...\\
					Capilaridad & $h = \dfrac{4 \sigma \cos \beta}{\gamma D}$
				\end{tabular}
			\end{multicols}
		
		También pensaba poner la ecuacion de los gases y algo de ese estilo que vimos en termo... pero no sé, qué opinan ustedes?
			
		\end{cajita}
	
	\unidad{2}{Estática de los fluidos}
		\begin{cajita}
			\centering
			\subtitulo{Fluidos en reposo}
			$dp = -\gamma dz$
			
			Agregar algo de manómetros estaría bien?
			
			\subtitulo{Fuerzas sobre áreas planas}
			
			\begin{tabular}{l l}
				Magnitud de F & $F = \gamma \bar{h} A = P_C A$\\
				Punto de aplicación C de F & $ y_P = \bar{y} + \dfrac{\bar{J}}{A \bar{y}}$\\
				$C:(x_P, y_P)$& $x_P = \bar{x} + \dfrac{\bar{J}_{xy}}{A \bar{y}}$\\
			\end{tabular}
		
			
		\begin{multicols}{2}
			\subtitulo{Flotabilidad}
			\begin{tabular}{l l}
				\multicolumn{2}{c}{$F_B = \gamma \vdes$}\\
				En equilibrio & $F=W$\\
			\end{tabular}
		
			\columnbreak
			
			\subtitulo{Estabilidad}
			
			$\overline{GM} = \dfrac{J_O}{\vdes} - \overline{CG}$
			
			
			\includegraphics[width=\linewidth]{estabilidad}
		\end{multicols}
		\end{cajita}
\end{document}