	\section{Introducción general}

\vspace{-.5cm}

\subsection{Conceptos básicos}

\subsubsection{Energía} La energía es una característica de la materia y puede transformarse o transferirse. Existen en la naturaleza de distintas maneras.

\subsubsection{Máquina} Es un dispositivo que transforma o transfiere la energía.

Las máquinas de fluidos utilizan \textsl{fluidos de trabajo} para generar energía y se clasifican en dos grandes grupos: \textsl{máquinas hidráulicas}, como las turbinas hidráulicas que transforman la energía cinética del agua en energía mecánica de rotación; y en \textsl{máquinas térmicas}, como las turbinas de vapor, donde el funcionamiento es similar a una turbina hidráulica exceptuando el cambio de propiedades que sufre el líquido de trabajo durante el transcurso.

\subsection{Clasificación de máquinas de los fluidos}
\lipsum[1]
\subsubsection{Máquinas Hidráulicas} Una máquina hidráulica utiliza como fluido de trabajo a los fluidos incompresibles o aquellos que se comporten como tal debido a que en el interior del sistema no sufren variaciones significativas en sus propiedades.
\subsubsection{Máquinas Térmicas} Es un dispositivo que transforma energía pero su fluido de trabajo cambia sus propiedades durante la operación de la máquina.

Se clasifican en dos: 
\begin{itemize}
	\item Turbomáquinas y;
	\item De desplazamiento positivo
\end{itemize}

En las \textsl{turbomáquinas} la sustancia de trabajo es impulsada a traves de un rotor provisto de palas y cuyo principio de funionamiento se basa en las ecuaciones de Euler. 	

\subsection{Aplicaciones de las Máquinas Térmicas}
\subsubsection{Turbinas de vapor}
\subsubsection{Turbinas de gas}
\subsubsection{Turbocompresores}
\subsubsection{Motoras}
\subsubsection{Generadoras}

\subsection{Ecuaciones de Euler}
\subsection{Principio de desplazamiento positivo}
\lipsum[2]