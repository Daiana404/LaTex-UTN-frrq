\documentclass[11pt,a4paper]{article}
\usepackage[utf8]{inputenc}
%\setmainfont{Butler}
\usepackage{amsmath}
\usepackage{amsfonts}
\usepackage{amssymb}
\usepackage{graphicx}
\usepackage{multicol}%Multicolumnas :)
\usepackage[left=2.00cm, right=2.00cm, top=3.00cm, bottom=3.00cm]{geometry}
\usepackage{fancyhdr}
\usepackage{tocloft} %Personalizar el formato del índice
\usepackage{nameref}%Lo uso para llamar a las secciones
\usepackage{textcase} %Hacer letras en mayúscula
\usepackage{lastpage} %Colocar el num de páginas
\usepackage{lipsum}
\usepackage[many]{tcolorbox}
\usepackage[explicit]{titlesec}
\renewcommand{\contentsname}{} % Elimina el título del índice
%BEGIN_FOLD SECTION / SUBSECTION / SUBSUBSECTION
	\newcommand{\aux}[1]{
		%	\begin{tabularx}{\textwidth} { 
				%			 >{\raggedright\arraybackslash}X 
				%			 >{\raggedleft\arraybackslash}X  }
			%		{\large  \thesection \ . \textsc{#1}} &  \thesection \\
			%	\end{tabularx}
		\newpage
		\textbf{\Huge #1} \\ {\small Unidad \thesection} \vspace{0.5cm}
		\label{sec:\thesection}
		\begin{multicols}{2} % Inicio de la sección de dos columnas
			\tableofcontents
		\end{multicols} % Fin de la sección de dos columnas
		
	}
	\titleformat{name=\section}
	{\normalfont}{}{0em}
	{\aux{#1}}[\addvspace{4ex}]
%END_FOLD
%BEGIN_FOLD CAJITAS
\newtcolorbox{preguntas}[1]{title = \textsc{\textbf{Pregunta de examen}},enhanced,attach boxed title to top left={xshift=2mm, yshift=-2mm}, colbacktitle=red!70!black, coltitle=white, boxed title style={colframe=red!30},colback=white!90!red, colframe=red!30}
%END_FOLD
%BEGIN_FOLD COMANDOS
	
%END_FOLD

%BEGIN_FOLD ENCABEZADO Y PIE
	\fancyhead[R]{\textbf{\nameref{sec:\thesection}}}
	\fancyhead[L]{\textsc{Máquinas Térmicas}}
	\fancyhead[C]{}
	\fancyfoot[R]{Página \thepage\ | \pageref{LastPage}}
	\fancyfoot[L]{UTN Facultad Regional Reconquista}
	\fancyfoot[C]{}
%END_FOLD

\begin{document}
	\pagestyle{fancy}
	\section{Introducción general}
	
	\subsection{Conceptos básicos}
	
	\subsubsection{Energía} La energía es una característica de la materia y puede transformarse o transferirse. Existen en la naturaleza de distintas maneras.
	
	\subsubsection{Máquina} Es un dispositivo que transforma o transfiere la energía.
	
	Las máquinas de fluidos utilizan \textsl{fluidos de trabajo} para generar energía y se clasifican en dos grandes grupos: \textsl{máquinas hidráulicas}, como las turbinas hidráulicas que transforman la energía cinética del agua en energía mecánica de rotación; y en \textsl{máquinas térmicas}, como las turbinas de vapor, donde el funcionamiento es similar a una turbina hidráulica exceptuando el cambio de propiedades que sufre el líquido de trabajo durante el transcurso.
	
	\subsection{Clasificación de máquinas de los fluidos}
	\lipsum[1]
	\subsubsection{Máquinas Hidráulicas} Una máquina hidráulica utiliza como fluido de trabajo a los fluidos incompresibles o aquellos que se comporten como tal debido a que en el interior del sistema no sufren variaciones significativas en sus propiedades.
	\subsubsection{Máquinas Térmicas} Es un dispositivo que transforma energía pero su fluido de trabajo cambia sus propiedades durante la operación de la máquina.
	
	Se clasifican en dos: 
	\begin{itemize}
		\item Turbomáquinas y;
		\item De desplazamiento positivo
	\end{itemize}
	
	En las \textsl{turbomáquinas} la sustancia de trabajo es impulsada a traves de un rotor provisto de palas y cuyo principio de funionamiento se basa en las ecuaciones de Euler. 	
	
	\subsection{Aplicaciones de las Máquinas Térmicas}
	\subsubsection{Turbinas de vapor}
	\subsubsection{Turbinas de gas}
	\subsubsection{Turbocompresores}
	\subsubsection{Motoras}
	\subsubsection{Generadoras}
	
	\subsection{Ecuaciones de Euler}
	\subsection{Principio de desplazamiento positivo}
	
	\section{Combustibles para calderas}
	\subsection{Combustibles}
	
	\begin{preguntas}
		. ¿Qué es un combustible?
	\end{preguntas}
	
	
	
\end{document}