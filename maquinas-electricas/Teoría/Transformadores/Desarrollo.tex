\fancyfoot{}
\rfoot{{\small Pág. \thepage}}
\lfoot{{\footnotesize \textbf{Alumno:} Melani Faulkner}}
\fancyhead{}
\lhead{{\footnotesize Mecánica de fluidos \\ \textit{UTN - Facultad Regional Reconquista}}}
\rhead{{\footnotesize \textsc{Laboratorio 1: Propiedades de fluidos}}}
\setlength{\headsep}{1.2cm}

\setlength{\parindent}{0pt}%para sacar la sangría



\section{Principio de funcionamiento de un transformador ideal}

Para el análisis se toma que el transformador está alimentado por el lado de alta (el de + espiras) y trabaja como un reductor. Es decir que:

1) El primario trabaja como un recepto respecto a la fuente. (recibe I)

2) El secundario se comporta como un generador respecto a la carga conectada a sus bordes. (Entrega I)

Para un transformador ideal no existen pérdidas por Histéresis y corrientes parásitas y no existen flujos de dispersión (todo el flujo magnético enlaza al primario y secundario).

\subsection{Transformador ideal sin carga}
\begin{figure}[!htbp]
	\centering
	\includegraphics[width=0.80\linewidth]{"../Figuras/1"}
\end{figure} 
 
 Realmente $e_{1}$ representa una f.c.e.m porque se opone a $v_{1}$ y limita la corriente de primario. La polaridad $e_{2}$ tiene en cuenta que, al cerrar el circuito,\footnote{Porque si está abierto, sólo hay V, \textbf{no I}} la corriente $i_{2}$ debe generar un flujo que se oponga el flujo primario que la originó. Es decir que, \textit{la f.m.m del secundario actúa en contra de la f.m.m primaria produciendo un efecto desmagnetizante sobre ésta.}
 
 Aplicando la 2° Ley de Kirchhoff al transformador ideal tenemos que:
 
 \begin{figure}[!htbp]
 	\centering
 	\includegraphics[width=0.40\linewidth]{"../Figuras/2"}
 	\label{fig:voltaje}
 \end{figure} 

Si se parte de un flujo \textbf{senoidal} de la forma:
 \begin{figure}[!htbp]
	\centering
	\includegraphics[width=0.40\linewidth]{"../Figuras/3"}
\end{figure} 

Haciendo la derivada del flujo y reemplazando en \ref{fig:voltaje} tenemos:
 \begin{figure}[!htbp]
	\centering
	\includegraphics[width=0.60\linewidth]{"../Figuras/4"}
\end{figure} 

Comparando el flujo expresado en coseno y los voltajes podemos ver que los últimos van 90° adelantados respecto al flujo. Si calculamos sus \textbf{valores eficaces}:

 \begin{figure}[!htbp]
	\centering
	\includegraphics[width=0.30\linewidth]{"../Figuras/5"}
\end{figure} 

Dividiendo entre sí las ecuaciones y simplificando resulta:

 \begin{figure}[H]
	\centering
	\includegraphics[width=0.20\linewidth]{"../Figuras/6"}
\end{figure} 

donde el factor \textit{m} se denomina \textbf{relación de transformación}.

Si el transformador está en \textbf{vacío} o \textbf{sin carga}, las pérdidas en el hierro $P_{Fe}$ en el núcleo del transformador será:

\begin{equation}
	P_{Fe} = V_{1}\,I_{0}\,cos\;\phi_{0}
\end{equation}

donde $V_{1}$ y $I_{0}$ representa los valores eficaces de la tensión y la corriente.

La corriente de vacío $I_{0}$ tiene dos componentes: una activa $I_{Fe}$ y una reactiva $I_{\mu}$

 \begin{figure}[!htbp]
	\centering
	\includegraphics[width=0.70\linewidth]{"../Figuras/7"}
\end{figure} 

\subsection{Transformador ideal con carga}

Si cerramos el interruptor S, el transformador funciona \textbf{en carga} y aparece una corriente $i_{2}$ que circula por el secundario.
 \begin{figure}[!htbp]
	\centering
	\includegraphics[width=0.30\linewidth]{"../Figuras/8"}
\end{figure} 

La corriente $\mathbf{I_{2}}$ se retrasa $\phi_{2}$ de la f.e.m $\mathbf{E_{2}}$

La corriente $i_{2}$ en el secundario produce una f.m.m. desmagnetizante $N_{2}i_{2}$ que se opone a la f.m.m primaria $N_{1}i_{0}$. Para que el flujo no se vea reducido por este efecto , en el primario se genera una corriente adicional primaria $i'_{2}$ con una f.m.m equivalente:

 \begin{figure}[H]
	\centering
	\includegraphics[width=0.18\linewidth]{"../Figuras/9"}
\end{figure} 

de donde se deduce el valor de la corriente $i'_{2}$ adicional primaria:

 \begin{figure}[H]
	\centering
	\includegraphics[width=0.30\linewidth]{"../Figuras/10"}
\end{figure} 

La corriente total necesaria en el primario $i_{1}$ será igual a:

 \begin{figure}[H]
	\centering
	\includegraphics[width=0.30\linewidth]{"../Figuras/11"}
\end{figure} 

La ecuación nos indica que la corriente primaria $i_{1}$ tiene dos componentes:

\begin{itemize}
	\item \textbf{Una corriente de excitación o de vacío} $\mathbf{I_{0}}$ que produce el flujo en el núcleo magnético y vence las pérdidas en el hierro a través de sus componentes $I_{\mu}$ y $I_{Fe}$.
	\item \textbf{Una componente de carga} $\mathbf{I'_{2}}$ que equilibra o contrarresta la acción desmagnetizante de la f.m.m secundaria para que el flujo en el núcleo permanezca constante e independiente de la carga. Esta se denomina \textbf{corriente secundaria reducida}.
\end{itemize}

A plena carga la corriente $\mathbf{I'_{2}}$ es 20 veces por lo menos mayor que $\mathbf{I_{0}}$ por lo que puede despreciarse y la ecuación queda:

 \begin{figure}[H]
	\centering
	\includegraphics[width=0.15\linewidth]{"../Figuras/12"}
\end{figure} 

\section{Funcionamiento de un transformador real}

En el análisis de un trafo real se tiene en cuenta la resistencias $R_{1}$ y $R_{2}$ de los arrollamientos y los flujos de dispersión $\Phi_{d1}$ y $\Phi_{d2}$ que se cierran en el aire.

Si consideramos los flujos de dispersión desaparece la idea del flujo común único que existía en el transformador ideal. Si tomamos que $\Phi_{1}$ y $\Phi_{2}$ son los flujos totales que atraviesan los devanados primario y secundario y $\Phi$ es el flujo común a ambos se cumplirá:

 \begin{figure}[H]
	\centering
	\includegraphics[width=0.3\linewidth]{"../Figuras/13"}
\end{figure} 

Para representar estas pérdidas agregamos las resistencias $R_{1}$ y $R_{2}$ y dos bobinas adicionales con núcleo de aire que representan los flujos de dispersión $\Phi_{d1}$ y $\Phi_{d2}$ donde se han indicado con $L_{d1}$ y $L_{d2}$ son los coeficientes de autoinducción, cuyos valores serán:

 \begin{figure}[H]
	\centering
	\includegraphics[width=0.3\linewidth]{"../Figuras/14"}
\end{figure} 

y que dan lugar a las reactancias de dispersión $X_{1}$ y $X_{2}$ de ambos devanados:
 \begin{figure}[H]
	\centering
	\includegraphics[width=0.3\linewidth]{"../Figuras/15"}
\end{figure} 

 \begin{figure}[H]
	\centering
	\includegraphics[width=0.8\linewidth]{"../Figuras/16"}
\end{figure} 

La aplicación del 2° Ley de Kirchhoff a los circuitos primario y secundario nos da:

 \begin{figure}[H]
	\centering
	\includegraphics[width=0.6\linewidth]{"../Figuras/17"}
\end{figure} 

Las ecuaciones para calcular $V_{1}$ y $V_{2}$ quedan:
\begin{equation}
	\mathbf{V_{1}=E_{1}+R_{1} \, I_{1}+ j \, X_{2} \, I_{2}} \hspace{5mm} ; \hspace{5mm} \mathbf{V_{2}=E_{2} - R_{2}\, I_{2} - j X_{2}\, I_{2}}
\end{equation}
 
Como se puede apreciar, las caídas de tensión $V_{1}$ y $V_{2}$ no son iguales a $E_{1}$ y $E_{2}$ por lo tanto la relación de transformación para trafos reales queda:
\begin{equation}
	\dfrac{E_{1}}{E_{2}}=\dfrac{N_{1}}{N_{2}}=m
\end{equation}
En los transformadores industriales la caída de tensión provocada por el cobre son muy pequeñas por lo que podemos decir:
\begin{equation}
	V_{1} \approx E_{1} \hspace{5mm}  V_{2} \approx E_{2}  \hspace{5mm} \therefore \hspace{5mm} \dfrac{V_{1}}{V_{2}} \approx m
\end{equation}

Si el transformador trabaja en \textbf{vacío}, $I_{2}=0$ por lo tanto las ecuaciones quedan:
\begin{equation}
	\mathbf{V_{1}=E_{1} + R_{1} \, I_{0} + jX_{1} \, I_{0}} \hspace{5mm} ; \hspace{5mm} \mathbf{V_{2}=E_{2}}
\end{equation}

Las caídas de tensiones en vacío son muy pequeñas por lo tanto:
\begin{equation}
	V_{1}=E_{1} \hspace{5mm} ; \hspace{5mm} V_{20}=E_{2} \hspace{5mm} \therefore \hspace{5mm} m=\dfrac{V_{1}}{V_{20}}=\dfrac{E_{1}}{E_{2}}=\dfrac{N_{1}}{N_{2}}
\end{equation}

Este cociente con la tensión secundaria en vacío es el que incluye el fabricante en la placa características de la máquina. La ecuación $\mathbf{I_{1}=I_{0}+\dfrac{I_{2}}{m}}$ es válida a todos los efectos.

\section{Circuito equivalente de un transformador}
Para el desarrollo de un circuito equivalente de un transformador se inicia \textbf{reduciendo} ambos devanados al mismo número de espiras. Si reducimos el secundario al primario, entonces $N_{1}=N'_{2}$. Para que este nuevo transformador sea equivalente al original \textit{deben conservarse las potencias activa y reactiva} y su distribución en los distintos elementos del circuito secundario. 
 \begin{figure}[H]
	\centering
	\includegraphics[width=0.8\linewidth]{"../Figuras/18"}
	\caption{Circuito equivalente de un transformador real reducido al primario}
\end{figure} 

\subsection*{a) F.e.m.s y tensiones}
Si $N'_{2}=N_{1} \; \therefore \; \dfrac{E_{1}}{E'_{2}}=\dfrac{N_{1}}{N'_{2}}=1 \Rightarrow \boxed{E'_{2}=E_{1}=m \, E_{2} \; \wedge \; V'_{2}=m \, V_{2}}$

\subsection*{b) Corrientes}
Como la potencia aparente en ambos secundarios se conserva $S_{2}=V_{2}\, I_{2}=V'_{2}\, I'_{2}$ reemplazando $V'_{2}=m \, V_{2}$ queda:
\begin{equation}
	I'_{2}=\dfrac{I_{2}}{m}
\end{equation}

\subsection*{c) Impedancias}
Como la potencia activa se conserva $R_{2}\, {I_{2}}^{2} = R'_{2} \, {I'_{2}}^{2}$ reemplazando $I'_{2}=\dfrac{I_{2}}{m}$ se tiene:
\begin{equation}
	R'_{2}=m^{2}\, R_{2}
\end{equation}
Lo mismo sucede con la potencia reactiva por lo tanto:
\begin{equation}
	X'_{2}=m^{2}\, X_{2}
\end{equation}
Entonces para una impedancia $Z'_{L}=m^{2}\, Z_{L}$\\

La importancia fundamental de la reducción de los devanados es obtener una representación del transformador donde no exista una relación de transformación porque $N_{1}=N'_{2}$ y sustituir los devanados acoplados magnéticamente por otros \textbf{acoplados sólo eléctricamente}.
%Me quedé pag 199 PDF/185 LIBRO