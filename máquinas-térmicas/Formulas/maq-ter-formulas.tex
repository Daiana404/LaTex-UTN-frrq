\documentclass[11pt,a4paper]{article}
\usepackage[utf8]{inputenc}
\usepackage[T1]{fontenc}
\usepackage{amsmath}
\usepackage{amssymb}
\usepackage{amsfonts}
\usepackage{graphicx}
\usepackage[spanish]{babel}
\usepackage{booktabs, fourier, tabularx, wrapfig, multicol,multirow,caption, subcaption,tikz, fancyhdr, steinmetz, xcolor}
\usepackage[many]{tcolorbox}

\usepackage[left=2cm,right=2cm,top=2cm,bottom=2cm]{geometry}

\graphicspath{{./img/}}

%Acá voy a probar la rama
%%Acá va la segunda parte

%Configuración y comandos

%tcolorbox personalizado
\newtcolorbox{cajita}{colback=white!97!brown, colframe=brown!15!gray, breakable}

%Encabezado y pie de página
\fancyhead[R]{\textsc{Máquinas Térmicas}}
\fancyhead[L]{\includegraphics[width=.1\textwidth]{utncom}}
\fancyhead[C]{Hoja de fórmulas}
\fancyfoot[R]{ \vspace{.1cm} Página \thepage }
\fancyfoot[L]{\hrule \vspace{.1cm} Ingeniería Electromecánica}
\fancyfoot[C]{ \vspace{.1cm}}

\newtcolorbox{mybox}[1]{title = \textsc{Unidad #1},colbacktitle=red!85!black,enhanced,attach boxed title to top center={yshift=-2mm}, colbacktitle=white, coltitle=gray!50!black, boxed title style={colframe=blue!30},colback=white!97!brown, colframe=blue!30}

%Comando para fasores :)
\newcommand{\fasor}[1]{\uppercase{\textit{\textbf{#1}}}}
%Comando título de cada unidad
\newcommand{\unidad}[2]{\begin{center}
		\fontsize{10}{10}\selectfont\color{gray!50!black}\scshape Unidad #1 \\
		\fontsize{14}{14}\selectfont \scshape #2
	\end{center} \vspace{-.6cm}}
%Comando pra subtitulos
\newcommand{\subtitulo}[1]{
	\textbf{#1} \\ \vspace{.1cm} {\color{gray} \hrule}
}

\begin{document}
	\pagestyle{fancy}
	\begin{cajita}
		\section*{Poder calorífico}
			\begin{center}
				\begin{tabular}{r l}
					Relación entre los poderes caloríficos: & \boxed{PCI = PCS - 597 \times G= PCS - 597(9H+H_{2}O)}\\
				\end{tabular}
			\end{center}
			\begin{multicols}{2}
			Siendo:\\
				\begin{tabular}{r p{0.4\textwidth}}
					PCI & poder calorífico inferior\\
					PCS & poder calorífico superior\\
					597	& Calor de condensación del agua a O ºC \\
					G & Porcentaje en peso del agua formada por la combustión del $H_{2}$ más la humedad propia del combustible\\
				\end{tabular}
			\newpage
			
			Recordando: \boxed{G=9H+H_{2}O} $\uparrow$\\
			
				\begin{tabular}{r p{0.4\textwidth}}
					9 & Son los kilos de agua que se forman al oxidar un kilo de  hidrógeno.\\
					H & \% de hidrógeno contenido en el combustible.\\
					H2O & \% de humedad del combustible.\\
				\end{tabular}
			\end{multicols}
		
	\begin{center}
		\subtitulo{ Método analítico}
	\end{center}
%\vspace{0.2cm}
	\begin{tabular}{l l}
		\multicolumn{2}{l}{\textbf{Formulas de Dulong}}\\
		PCS comb. seco & \boxed{PCS	=	8.140  \times	C	+ 34.400	\times	(H -	O/8)	+	2.220	\times	S}\\
		PCI comb. seco: &\boxed{PCI = 8.140 \times C + 29.000 \times (H - O/ 8 ) + 2.220 \times S}\\
		PCI comb. húmedo:&\boxed{PCI = 8.140 \times C + 29.000 \times (H - O/ 8 ) +  2.220 \times S - 600 \times H2O}\\[0.2cm]
		\multicolumn{2}{l}{\textbf{Formula de Hutte}}\\
		PCI comb. húmedo & \boxed{8.100 \times C + 29.000 \times (H - O/ 8 ) + 2.500 \times S -  600 \times H2O}\\[0.2cm]
		\multicolumn{2}{l}{\textbf{Formula de Asociación de Ing. Alemanes}}\\
		PCI comb. húmedo& \boxed{PCI = 8.080 \times C + 29.000 \times (H - O/ 8 ) + 2.500 \times S -  600 \times H2O}\\
	\end{tabular}
	\begin{flushleft}
		\begin{tabular}{r p{0.87\textwidth}}
				C & Cantidad centesimal de carbono en peso por kilogramo combustible\\
				H & Cantidad centesimal de hidrógeno total en peso por kilogramo de
				combustible\\
				O & Cantidad centesimal de oxígeno en peso por kilogramo combustible\\
				S & Cantidad centesimal de azufre en peso por kilogramo combustible\\
				O / 8 & Cantidad centesimal de hidrógeno en peso que se encuentra combinado con el oxígeno del mismo combustible dando ``agua de  combinación''\\
				(H - O/8 ) &  Cantidad centesimal de ``hidrógeno disponible'', en peso  realmente disponible para que se oxide con el oxígeno del aire,  dando ``agua de formación''\\
			\end{tabular}
		\end{flushleft}

	\begin{center}
		\subtitulo{ Método práctico}		
	\end{center}
		CALORIMETRO DE MAHLER Y KROEKER
			\begin{flushleft}
			\begin{tabular}{r p{0.85\textwidth}}
				$Q$=&$Q_{agua}+Q_{termometro}+Q_{agitador}+Q_{recipiente}+Q_{vaso}$\\
				$Q$=&$\Delta T (m_{agua}~cp_{agua}+m_{termometro}~cp_{termometro}+m_{agitador}+cp_{agitador}+m_{recipiente}~cp_{recipiente}+m_{vaso}~cp_{vaso})$\\
				$Q$&$\left(m_{agua}~cp_{agua}+E_{aparato}\right)\Delta T$\\					
				\multicolumn{2}{l}{	Para determinar el poder calorifico:}\\
				$Q$=&$Q_{combustible}+Q_{alambre}$\\
				$Q_{comb}$=&$Q-Q_{alambre}$\\
				\multicolumn{2}{l}{Reemplazo:}\\
				$Q_{comb}$=&$\left(m_{agua}~cp_{agua}+E_{aparato}\right)\Delta T - m_{alambre} ~ C_{alambre}$ \\
				\multicolumn{2}{l}{Nos queda:}\\
				PCS=&$\dfrac{Q_{combustible}}{G_{combustible}}$\\
				PCI=& $PCS-600	( 9H	+ H2O )$=$ PCS	-	600	\dfrac{G_{agua}}{G_{combustible}}$\\
			\end{tabular}\\
		\textit{$G_{agua}$ representa el peso del total de agua existente = (peso papel humedo - peso papel seco)\\
			    $G_{combustible}$ el peso de combustible quemado}
		\end{flushleft}
	
	
			
	\end{cajita}
	
\end{document}