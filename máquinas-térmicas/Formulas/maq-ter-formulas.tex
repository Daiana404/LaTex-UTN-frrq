\documentclass[11pt,a4paper]{article}
\usepackage{{../../paquete-formulas}}
\usepackage{{../../estilos-formulas}}
\usepackage[thinlines]{easytable} %Para tener comando TAB y tablas dif.


\newcommand{\materia}{Máquinas Térmicas}
\newcommand{\vs}{\vspace{-.3cm}}


\begin{document}
	\pagestyle{pieyencabezado}
	
	\section*{Nomenclatura}
	\begin{tabular}{r l}
		PCI $[kcal / kg_{comb}]$ & Poder calorífico inferior \\
		PCS $[kcal / kg_{comb}]$ & Poder calorífico superior \\
		$H,\ S,\ C,\ O$ & Composición gravimétrica respectiva al elemento (cant. centesimal)\\
		$g$ & Composición gravimétrica (cant. centesimal)\\
		$r$ & Composición volumétrica (cant. centesimal)\\
		$H_2O$ & \% de humedad en el combustible \\
		$G [kg]$ & Peso\\
		m & Masa\\
		c$_{L}$ & Calor latente\\
		c$_p$ & Calor específico\\
		1 $m^3$ & 1 $tn$ de vapor
	\end{tabular}

	\unidad{2}{Combustión}
	
	\begin{multicols}{2}
		\begin{cajita}
			
			\subtitulo{Poder calorífico}
			
			\vspace{.2cm}
			
			\includegraphics[width = \linewidth]{U2-poderes-calorificos}\vs
			
			\begin{flushleft}
				Relación entre los poderes caloríficos
			\end{flushleft}\vspace{-.2cm}
			
			$PCI = PCS - Q_{vapor} = PCS - 579 G$\vspace{.2cm}
			
			
			\boxed{PCI = PCS - 579 \left(9\ H+H_2O\right)}\vspace{.2cm}
			
			\begin{tabular}{r p{.7\linewidth}}
				$Q_{vapor}$ & Calor de condensación del vapor de agua\\
				$G$ & \% en peso del agua formada por la combustión más la humedad del combustible.\\
				$597$ & Calor de condensación del agua a $0^\circ C$.
			\end{tabular}
			
			\subsubtitulo{Hidrógeno}
			\vspace{-.8cm}
			
			
			\begin{flushleft}
				Reacción química de la combustión completa del hidrógeno
			\end{flushleft}\vs
			
			$H_2 + \frac{1}{2}O_2 \rightarrow H_2O + \boxed{34400} \left[\frac{kcal}{kg_H}\right]$
		
		\subsubtitulo{Carbono}
		\vspace{-.8cm}
		
		\begin{flushleft}
			Reacción química de la combustión completa del carbono
		\end{flushleft}\vs
	
		$C + O_2 \rightarrow CO_2 + \boxed{8140} \left[\frac{kcal}{kg_C}\right]$
		
		\begin{flushleft}
			Reacción química de la combustión incompleta del carbono
		\end{flushleft}\vs
	
		$C + \frac{1}{2} O_2 \rightarrow CO + \boxed{2440} \left[\frac{kcal}{kg_C}\right]$
			
		\subsubtitulo{Azufre}
		\vspace{-.8cm}
		
		\begin{flushleft}
			Reacción química de la combustión para el azufre.
		\end{flushleft}\vs
	
		$S + O_2 \rightarrow SO_2 + \boxed{2220} \left[\frac{kcal}{kg_S}\right] $
		
		\end{cajita}
		\columnbreak
		\begin{cajita}
			\subtitulo{PC Método analítico}
			
			\subsubtitulo{Fórmula de Dulong}
			\vspace{-.8cm}
			
			\begin{flushleft}
				PC de un combustible seco
			\end{flushleft}
			
			\vspace{-.2cm}
		
			\boxed{PCS = PCI = 8140\  C + 34400 \left(H - \dfrac{O}{8} \right) + 2220\ S}
			
			\begin{flushleft}
				PCI de un combustible húmedo
			\end{flushleft}
			
			\vspace{-.2cm}
			
			\boxed{PCI = 8140\  C + 34400 \left(H - \dfrac{O}{8} \right) + 2220\ S - 600\ H_2O}
			
			\subsubtitulo{Fórmula de Hutte}
			
			\vspace{-.8cm}
			
			\begin{flushleft}
				PCS de un combustible húmedo
			\end{flushleft}
		
			\vspace{-.2cm}
			
			\boxed{PCI = 8100\  C + 29000 \left(H - \dfrac{O}{8} \right) + 2500\ S - 600\ H_2O}
			
			\subsubtitulo{Fórmula de la Asociación de Ingenieros Alemanes}
			
			\vspace{-.8cm}
			
			\begin{flushleft}
				PCI de un combustible húmedo
			\end{flushleft}
			
			\vspace{-.2cm}
			
			\boxed{PCI = 8080\  C + 29000 \left(H - \dfrac{O}{8} \right) + 2500\ S - 600\ H_2O}
			\vspace{.2cm}
			
			\begin{tabular}{r p{.7\linewidth}}
				$\dfrac{O}{8}$ & \% de $H_2$ en peso combinado con el $O_2$ del combustible dando \emph{agua de combinación}\\
				$H - \dfrac{O}{8}$ & \% de \emph{hidrógeno disponible} en peso que se oxida con el aire ($O_2$) para dar \emph{agua de formación}
			\end{tabular}
		\end{cajita}
	\end{multicols}
	
	\begin{cajita}
		
		\subtitulo{PC Método práctico}
		
		\subsubtitulo{Calorímetro de Mahler y Kroeker}\vspace{-.6cm}
		
		\begin{multicols}{2}
			\begin{flushleft}
				Supone que el calor $Q$ generado dentro de la bomba calorimétrica es absorbido por los elementos que la rodean:
				\begin{itemize}
					\item Agua contenida ($w$)\\[.1cm]
					\item Agitador\\[.1cm]
					\item Termómetro\\[.1cm]
					\item Bomba\\[.1cm]
					\item Recipiente
				\end{itemize} 
			
				Y dicho calor es cedido por la combustión y el alambre:
			\end{flushleft} 
			\begin{tabular}{r c l}
				$Q$ & = & $Q_{combustible} + Q_{alambre}$\\[.2cm]
					& = & $\left(m_{w} {c_p}_{w} + E_{aparato}\right) \Delta t$\\[.2cm]
			\end{tabular}
			
			\columnbreak
			
			$PCS = \dfrac{Q_{comb}}{G_{comb}}$\\[.1cm]
			
			
			\boxed{PCS = \dfrac{\left(m_{w} {c_p}_{w} + E_{aparato}\right) \Delta t - m_{alam} {c_L}_{alam}}{G_{comb}}}\\[.1cm]
			\boxed{PCI = PCS - 600 \dfrac{G_w}{G_{comb}}}
			\vspace{.3cm}
		
			
			\begin{tabular}{r p{.7\linewidth}}
				$G_w$ & Peso total de agua existente $= papel\,húmedo - papel\, seco$\\
				$G_{comb}$ & Peso de combustible quemado\\
			\end{tabular}
		\end{multicols}
		
	\end{cajita}
	
	\begin{cajita}
			
		\begin{center}
			\subtitulo{Aire mínimo}
		\end{center}
	
		\begin{multicols}{2}
			\subsubtitulo{Composición del aire atmosférico}
			
			\begin{tabular}{r l}
				En volumen & 21\% O$_2$\\
				& 79\% N$_2$\\
				En peso    & 23\% O$_2$\\
				& 77\% N$_2$\\
			\end{tabular}
		
			\subsubtitulo{Hidrógeno disponible}
			
			\boxed{Hd = H - \dfrac{O_2}{8}}
			
			
			\subsubtitulo{Exceso de aire}\vs
			
			\begin{flushleft}
				En la práctica se trabaja con un exceso de aire de 3-4\%  para una combustión completa.
			\end{flushleft}
		
			\boxed{V_{exc} = V_{at} \left( 1 + e \right)}
			
			\begin{flushleft}
				$e$ \, Porcentaje de exceso de aire
			\end{flushleft}
		
			\columnbreak
			
			\subsubtitulo{Cantidad teórica o mínima de aire}
			
			\boxed{V_{at} = 8.89\ C + 26.27\ Hd + 3.34\ S}\\ \vspace{.3cm}
			
			\boxed{G_{at} = 11.6\ C + 34.78\ Hd +4.35\ S}
		
			\begin{flushleft}
				Donde: \vspace{.5cm}
				
				\begin{tabular}{r p{.5\linewidth}}
					 $V_{at} \left[\dfrac{m^3_{aire}}{kg_{comb}}\right]$ & Porcentaje en volumen del aire teórico \\
					 $G_{at} \left[\dfrac{kg_{aire}}{kg_{comb}}\right]$ & Porcentaje en peso del aire teórico
				\end{tabular}
			\end{flushleft}
			
		\end{multicols}
		
	\end{cajita}
		\begin{cajita}
		
		\subtitulo{Gases de combustión}
		
		\vspace{.5cm}
		
		\boxed{g_{humo} = (3.67\ C + 9\ Hd +2\ S) + 3.35\ (2.67\ C + 8\ Hd + S) + g_{agua}}\\\vspace{.3cm}
		
		\boxed{V_{humo} = 1.897\ C + 11.2\ Hd +0.7\ S + 3.76\ (1.867\ C + 5.6\ Hd +0.7\ S) + 1.24\ g_{agua}}\\\vspace{.2cm}
		
		\begin{tabular}{r l}
			$g_{humo}$ & (kg de humo / kg de combustible)\\[.2cm]
			$V_{humo}$ & (m$^3$ de humo / kg de combustible)\\[.2cm]
			$g_{agua}$ & (kg de agua / kg de combustible)\\
		\end{tabular}
	
	\end{cajita}
	
	\begin{cajita}
		\subtitulo{Exceso de aire}
		
		\begin{multicols}{2}
				\boxed{g_{hum} = 1 + e\ G_{at}}\\[.2cm]
				
				\boxed{G_{at} = 11.6\ C + 37.38\ Hd + 4.35\ S}\\[.2cm]
				
				\boxed{g_{hum} = g_{sec} + 9\ H}\\[.2cm]
				
				\boxed{g_{sec} = C\ g_{sec/C} = C\ \left(\dfrac{\sum \mu_i\ r_i}{\mu_{CO_2}\ r_{CO_2}}\right)}\\[.2cm]
				
				\columnbreak
				
				\begin{tabular}{r p{.9\linewidth}}
					$e$ & Porcentaje de exceso de aire\\
					$\mu$ & Peso molecular (kg / kmol)\\[.1cm]
					$9\ H$ & \textbf{Agua de formación} (kg de agua / kg de combustible)\\
					$g_{hum}$ & (kg de gases húmedos / kg de combustible)\\
					$g_{sec}$ & (kg de gases secos / kg de combustible)\\
					$g_{sec/C}$ & (kg de gases secos / kg de carbono)\\
				\end{tabular}
		\end{multicols}
	
		
		\begin{multicols}{2}
			Desarrollo para llegar a $g_{sec}$
			\begin{gather*}
				g_{sec/C} = \dfrac{g''_{sec}}{g'_{C}}\\
				g''_{sec} = \sum_{i=1}^{n} g_i = \sum_{i=1}^{n} \mu_i~r_i\\
				g'_{C} = \sum_{i=1}^{n} \mu_{iC}~r_{iC}
			\end{gather*}
		
			\begin{tabular}{r p{0.87\textwidth}}
				$g''_{sec}$ & (kg gases secos / kmol combustible)\\
				$g'_{C}$ & (kg carbono / kmol combustible)\\
			\end{tabular}
		
		\end{multicols}
	
	\end{cajita}
		
	% \newpage
	\begin{cajita}
		\subtitulo{ Caracteristica de una caldera}
		
		\begin{multicols}{2}
			% \begin{tabular}{l{1cm} l{5cm}}
			% asdasdads& asdasdasd\\			
			% \end{tabular}
			\begin{tabular}{r  p{7cm} }
				$G_{v}$ & Vapor producido (kg / h)\\
				$R_{e}$& Relacion esteq. 9,7 para gas natural\\
				$\lambda$& Exceso de aire $\dfrac{21}{21-O_{2}}$\\
				$v$ & Velocidad gas\\
				$A_{2},\ B$& parametros dependen gas/caldera?\\
				$P_{ab}$ & va en MCA\\
				$P_C$& en combustible sin quemar es la potencia calorifica del carbono, 8140 $kcal/kg$\\
				$\% C$ & es Carbono sin quemar, en las cenizas\\
				$C_{p~humo}$ & se puede aproximar 0.24 $\frac{kcal}{kg~^{\circ} C}$\\
				$C_{v}$ & vapor sobrecalentado  0.46 $\frac{kcal}{kg~^{\circ} C}$
			\end{tabular}
		\vspace{0.4cm}
	
			\renewcommand{\arraystretch}{1.5}
	
			\textbf{PERDIDAS}\\
	
			\begin{tabular}{p{0.19\textwidth} l}
				Gases combustión & $Q = G_{g} {c_{p}}_{humo} (T_{gas} - T_{aire})$\\
				Humedad de comb. & $ Q = G_{H_2O}\ \Delta h_{H_2O}$\\
				Comb. sin quemar & $ Q = \dfrac{G_{escoria}}{G_{comb.}}\ PC\ C$\\
				Humedad comb. durante combustión & $Q = 9\ G_{hum} ~\Delta h_{agua}$\\
				Exceso aire & $Q = G_{aire}~c_{p~\textit{humo}} (T_{gas} - T_{aire})$\\
				Humedad aire & $ Q = G_{aire}~ c_{v} ~ (T_{gas} - T_{aire})$\\
				Pérdidas por radiación & \\
				Pérdidas circunstanciales & \\
			\end{tabular}
			
			\renewcommand{\arraystretch}{1.5}
		
			\begin{tabular}{p{0.18\textwidth} l}
				Calor absorbido por agua & $Q=\dfrac{G_v\,\Delta h_{agua}}{G_{comb}}$\\
				Capacidad  & $Q=G_{r} \, \Delta h$\\
				Potencia  (HP) & $P=\dfrac{G_{v}\,\left(\Delta h\right)}{543.4(kcal/kg)~~15.66(kg)}~~$\\
				&$~~=\dfrac{G_{v}\,\left(\Delta h\right)}{8510 (kcal/hp~h)}$\\[0.4cm]
				Aire combustion& $A_{c}=G_{c}\,Re\,\lambda$\\
				% \multicolumn{2}{r}{\begin{itemize}
				% 	\item ASD
				% \end{itemize}}\\
				\multicolumn{2}{c}{\textsl{Ejercicio dimensionamieto?}}\\
				Diametro tubo gas & $D=\sqrt{\dfrac{365.35\,G_{c}}{v~~P_{ab}}}~~$ \textit{(v<40m/s)}\\
				Presión abs gas& $P_{ab}=P_{atm}+P_{carga}+P_{contra}$\\
				Perdida gases & $pg=(T_{gas}-T_{amb})\left(\dfrac{A_{2}}{21-O_{2}}+B\right)$\\
				Rendimiento  & $ \eta=100-pg$\\
				Calor quemador& $Q=\dfrac{G_{v}(\Delta H)}{\eta_{caldera} \, n_{quemadores}}$\\
			\end{tabular}
		\end{multicols}
	\end{cajita}

	\begin{cajita}
			\begin{center}
				\textbf{Intercambiador (superficies de intercambio)}
			\end{center}
	
			\begin{multicols}{2}
				\renewcommand{\arraystretch}{1}
				\begin{tabular}{@{}p{1cm}@{}@{}p{7cm}@{}}
					$S$ & superficie de intercambio\\
					$n$ & número de tubos\\
					$l$ & longitud tubos\\
					$d$& diametro tubos\\
					$K$ & coef total de transmisión de calor\\
					$\alpha$ & coeficientes peliculares \break (humo/vapor/agua)\\
				\end{tabular}
				
				\renewcommand{\arraystretch}{1.5}
				\begin{tabular}{l l}
					$S=n~l~\pi~d$ &	$Q=S~K~\Delta T_m$\\[0.2cm]
					$\Delta T_m = \dfrac {\Delta T_1 - \Delta T_2}{ln\left(\frac{\Delta T_1}{\Delta T_2}\right)}$ & $K=\dfrac{1}{\dfrac{1}{\alpha_1}+\dfrac{e}{\lambda}+\dfrac{1}{\alpha_2}}$\\
				\end{tabular}\\
				\renewcommand{\arraystretch}{1}
				\begin{tabular}{@{}p{1cm}@{}@{}p{7cm}@{}}
					$e$& espesor tubo\\
					$\Delta T_m$ & diferencia log de temp\\
					$\lambda$ & conductividad térmica tubo\\
	
				\end{tabular}
			\end{multicols}
		\end{cajita}

\end{document}

%\begin{cajita}
%	\subtitulo{ Caracteristica de una caldera}
%	\renewcommand{\arraystretch}{1.25}
%	\begin{multicols}{2}
%		% \begin{tabular}{l{1cm} l{5cm}}
%		% asdasdads& asdasdasd\\			
%		% \end{tabular}
%		\begin{tabular}{@{}p{1cm}@{}@{}p{7cm}@{}}
%			$G_{r}$ & cantidad vapor producido\\
%
%			$R_{e}$&relacion estequio... 9,7 para gas natural\\
%			$\lambda$&exceso de aire $\dfrac{21}{21-O_{2}}$\\
%			$v$ & velocidad gas\\
%			$A_{2}$y$B$& parametros dependen gas/caldera?\\
%			$P_{ab}$ & va en MCA\\
%			$P_C$& en combustible sin quemar es la potencia calorifica del carbono, 8140 $kcal/kg$\\
%			$\% C$ & es Carbono sin quemar, en las cenizas\\
%		\end{tabular}
%
%		\renewcommand{\arraystretch}{1.5}
%
%		PERDIDAS\\
%
%		\begin{tabular}{r l}
%			gases comb & $Q = G_{g} c_{p~\textit{humo}} (t_{gas}-t_{aire})$\\
%			humedad comb & $ Q= G_{h2o} \Delta h_{agua}$\\
%			comb sin quemar & $ Q =\dfrac{G_{escoria}}{G_{combustible}}~P_{C}~\% C$\\
%			hw comb comb &$ Q=9~G_{humedad} ~\Delta h_{agua}$\\
%			exceso aire & $Q = G_{aire}~c_{p~\textit{humo}} (t_{gas}-t_{aire})$\\
%		\end{tabular}
%		
%		\renewcommand{\arraystretch}{1.5}
%	
%		\begin{tabular}{r l}
%			Capacidad  & $Q=G_{r} \, \Delta h$\\
%			Potencia  (HP) & $P=\dfrac{G_{v}\,\left(\Delta h\right)}{543.4(kcal/kg)~~15.66(kg)}~~$\\
%			&$~~=\dfrac{G_{v}\,\left(\Delta h\right)}{8510 (kcal/hp~h)}$\\[0.4cm]
%			Aire combustion& $A_{c}=G_{c}\,Re\,\lambda$\\
%			% \multicolumn{2}{r}{\begin{itemize}
%			% 	\item ASD
%			% \end{itemize}}\\
%			\multicolumn{2}{c}{\textsl{Ejercicio dimensionamieto?}}\\
%			Diametro tubo gas & $D=\sqrt{\dfrac{365.35\,G_{c}}{v~~P_{ab}}}~~$ \textit{(v<40m/s)}\\
%			Presión abs gas& $P_{ab}=P_{atm}+P_{carga}+P_{contra}$\\
%			Perdida gases & $pg=(T_{gas}-T_{amb})\left(\dfrac{A_{2}}{21-O_{2}}+B\right)$\\
%			Rendimiento  & $ \eta=100-pg$\\
%			Calor quemador& $Q=\dfrac{G_{v}(\Delta H)}{\eta_{caldera} \, n_{quemadores}}$\\
%		\end{tabular}
%	\end{multicols}
%
%		\begin{center}
%			\textbf{Intercambiador (superficies de intercambio)}
%		\end{center}
%
%		\begin{multicols}{2}
%			\renewcommand{\arraystretch}{1}
%			\begin{tabular}{@{}p{1cm}@{}@{}p{7cm}@{}}
%				$S$ & superficie de intercambio\\
%				$n$ & número de tubos\\
%				$l$ & longitud tubos\\
%				$d$& diametro tubos\\
%				$K$ & coef total de transmisión de calor\\
%				$\alpha$ & coeficientes peliculares \break (humo/vapor/agua)\\
%			\end{tabular}
%			
%			\renewcommand{\arraystretch}{1.5}
%			\begin{tabular}{l l}
%				$S=n~l~\pi~d$ &	$Q=S~K~\Delta T_m$\\[0.2cm]
%				$\Delta T_m = \dfrac {\Delta T_1 - \Delta T_2}{ln\left(\frac{\Delta T_1}{\Delta T_2}\right)}$ & $K=\dfrac{1}{\dfrac{1}{\alpha_1}+\dfrac{e}{\lambda}+\dfrac{1}{\alpha_2}}$\\
%			\end{tabular}\\
%			\renewcommand{\arraystretch}{1}
%			\begin{tabular}{@{}p{1cm}@{}@{}p{7cm}@{}}
%				$e$& espesor tubo\\
%				$\Delta T_m$ & diferencia log de temp\\
%				$\lambda$ & conductividad térmica tubo\\
%
%			\end{tabular}
%		\end{multicols}