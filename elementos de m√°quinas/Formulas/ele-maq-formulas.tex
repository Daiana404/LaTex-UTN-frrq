\documentclass[11pt,a4paper]{article}
\usepackage{{../../paquete-formulas}}
\usepackage{{../../estilos-formulas}}

\usepackage{enumitem}

\newcommand{\materia}{Elementos de Máquinas}

\begin{document}
\pagestyle{pieyencabezado}
% \setlength\itemsep{-3mm}

\begin{multicols}{2}
	\section*{Nomenclatura}
	\begin{tabular}{l r}
		$\sigma_{T}$&tensión tracción \\
		$\sigma_{C}$&tensión compresión\\
		$\sigma_{f}$&tensión fluencia\\
		$\sigma_{adm}$&tensión admisible\\
		$Cs$&Coeficiente de seguridad\\
	\end{tabular}
	
	\begin{cajita}
		\section*{5 teorías de rotura}
		\begin{flushleft}
			1- Máxima tensión corte (GUEST):
			\begin{equation*}
				\dfrac{\sigma_{T}}{C_{s}}=\sigma_{adm}\geq\sqrt{(\sigma_{x}-\sigma_{y})^{2}+4~\tau_{xy}^{2}}
			\end{equation*}
			2- Máxima tensión normal:
			\begin{equation*}
				\dfrac{\sigma_{RT}}{C_{s}}=\sigma_{adm}\geq\dfrac{\sigma_{x}+\sigma_{y}}{2}+\dfrac{1}{2}\sqrt{(\sigma_{x}-\sigma_{y})^{2}+4~\tau_{xy}^{2}}
			\end{equation*}
			3- Deformaciones principales:
			\begin{equation*}
				\dfrac{\sigma_{RT}}{C_{s}};\dfrac{\sigma_{RC}}{C_{s}}=\sigma_{adm}
			\end{equation*}
			\begin{equation*}
				\sigma_{adm}\geq(\sigma_{x}-\sigma_{y})\left(\dfrac{1-\mu}{2}\right)\pm\dfrac{1+\mu}{2}\sqrt{(\sigma_{x}-\sigma_{y})^{2}+4~\tau_{xy}^{2}}
			\end{equation*}
			4- Energía deformación
			\begin{equation*}
				\sigma_{adm}=\dfrac{\sigma_{f}}{Cs}\geq\sqrt{\sigma_{x}^{2}+\sigma_{y}^{2}-2\mu \sigma_{x} \sigma_{y} +2(1+\mu)\tau_{xy}^{2}}
			\end{equation*}
			5- Energía distorión:
			\begin{equation*}
				\sigma_{adm}=\dfrac{\sigma_{f}}{Cs}\geq\sqrt{\sigma_{x}^{2}-\sigma_{x} \sigma_{y}+\sigma_{y}^{2}+3\tau_{xy}^{2}}			
			\end{equation*}
		\end{flushleft}
		\begin{tabular}{l r}
			(1,4,5 $\rightarrow$ material dúctil) & (2,3 $\rightarrow$ material frágil)
		\end{tabular}
		
	\end{cajita}

	\begin{cajita}
		\section*{Deformaciones}
		\begin{center}
			$e=\dfrac{l-l_{0}}{l_{0}}$
		\end{center}
		\begin{itemize}[itemsep=-2mm]
			\item $e$ = alargamiento especifico.
			\item $l$ = longitud "final"
			\item $l_{0}$ = longitud inicial
		\end{itemize}
	\end{cajita}
\end{multicols}




\end{document}