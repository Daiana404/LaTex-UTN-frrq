\documentclass[11pt,a4paper]{article}
\usepackage{{../../paquete-formulas}}
\usepackage{{../../estilos-formulas}}

\newcommand{\materia}{Electrónica Industrial}

\begin{document}
	\pagestyle{pieyencabezado}
	\section*{Nomenclatura}
	
	\begin{tabular}{r l r l}
		$V_Z$ & jeje &&\\
	\end{tabular}
	\unidad{1}{Dispositivos de estado sólido}
	\begin{multicols}{2}
		\begin{cajita}
			hola jaja
		\end{cajita}
		
		
		\begin{cajita}
			chau jeje
		\end{cajita}
	\end{multicols}

	\begin{cajita}
		buen día
	\end{cajita}

	\unidad{2}{Transistores}
	\begin{multicols}{2}
		\begin{cajita}
			\subtitulo{Polarización del BJT}
			
			\subsubtitulo{Ecuaciones del dispositivo}
			
			\begin{tabular}{c c}
				$i_C = \alpha i_E$ & $i_C = \beta i_B$ \\[.1cm]
				\multicolumn{2}{l}{Si no se especifica, $\alpha = 1 $}\\[.1cm]
				$V_{BB} = V_{R_B} + v_{BE}$ & $V_{CC} = V_{R_C} + v_{CE}$ \\[.1cm]
				\multicolumn{2}{c}{$i_E = i_B + i_C$}
			\end{tabular}
		
		\subsubtitulo{Aplicación en conmutación}\vspace{-.3cm}
		
		Garantizar que: $\beta i_B = 5 i_C$\\
		
		
		\begin{tabular}{l l}
			\textbf{Corte} & \textbf{Saturación} \\[.1cm]
			$i_B = 0$ & $v_{CE} = 0.2 V$ \\[.1cm]
			$i_C = i_{fuga}$ & $i_C = \dfrac{V_{CC}}{R_C + R_E}$\\[.1cm]
			Interruptor abierto & $v_{CE} = V_{CC}$ \\[.1cm]
			& Interruptor cerrado \\
		\end{tabular}
		
		\subsubtitulo{Aplicación para amplificación}\vspace{-.8cm}
		\begin{flushleft}
			Condición para aplicar el método aproximado:
		\end{flushleft}
			
			$\beta R_E \geq 10 R_2$\\
		\end{cajita}
	
	
		\begin{cajita}
			chau jeje
		\end{cajita}
	\end{multicols}
\end{document}