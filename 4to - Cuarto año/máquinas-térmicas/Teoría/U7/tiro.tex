\section{Tiro y Equipos de Recuperación}

\subsection{Tiro natural y artificial}

El fenómeno que se conoce como \emph{tiro} permite a una instalación generadora de vapor, que utiliza combustible industrial, la circulación de los gases de combustión a través de los distintos circuitos del sistema, la descarga de estos humos a la atmósfera, y la inyección al hogar del aire necesario para la combustión.

El tiro se puede lograr de forma \emph{natural} o \emph{artificial}.

\subsubsection{Tiro natural}

El tiro natural es consecuencia de la tendencia que tiene toda masa de gases calientes a ascender con respecto a otras más frías. La aspiración necesaria para que los gases puedan vencer la resistencia (pérdida de carga) que encuentran en su trayecto a través de la instalación con cierta velocidad, y ser expulsado al exterior, lo provoca una \textbf{chimenea}.


Para que el fenómeno se lleve a cabo debe existir una \textbf{diferencia de densidades} entre los gases calientes en el interior de la chimenea y el aire más frío que la rodea. 

La aspiración de los humos representa, de esta manera, el \emph{tiro estático} $h_a$, cuyo valor se lo puede calcular como:

\begin{equation*}
	h_a = \Delta P = P_{ext} - P_{int} \left[\dfrac{kg}{cm^2}\right]
\end{equation*}


Debido a que el valor del tiro natural generalmente es muy pequeño, a este se lo expresa en milímetro de columna de agua ($mm.c.a$). Entonces, se tiene que:

\begin{equation*}
	H[mm.c.a] = H_C g (\rho_{aire} - \rho_{gases})
\end{equation*}
\begin{equation}
	H[mm.c.a] = H_C (\gamma_{aire} - \gamma_{gases})
\end{equation}

donde $\gamma$ representa el peso específico en $\left[\dfrac{kgf}{cm^2}\right]$ y $H_C [m]$ la altura vertical de la chimenea.

El inconveniente del tiro natural es que para lograr la suficiente depresión obliga a que la chimenea tenga una gran altura, aproximadamente de 60-70m.

\subsubsection{Tiro artificial}

Cuando la circulación de los humos se logra por otro medio además de la implementación de una chimenea, se dice que el tiro es artificial. Según las condiciones, el tiro artificial puede ser:
\begin{itemize}
	\item Forzado, es cuando se inyecta aire al hogar.
	\item Inducido, cuando los gases de combustión son aspirados lo que provoca una depresión en el hogar.
	\item Equilibrado, cuando el tiraje es una combinación de los dos casos anteriores y se considera que el hogar está en estado de equilibrio.
\end{itemize}

En la práctica es conveniente que en el hogar exista una depresión de $3-4\ mm.c.a.$

En cualquiera de los tres sistemas se debe tener presente que la presión en el hogar puede ser positiva, cuando es mayor que la atmosférica; negativa, cuando es menor; y equilibrada, cuando es prácticamente igual a la atmosférica.\\


Si la presión es positiva puede ocurrir que se produzca un retroceso de llama por las puertas del hogar y eventuales fisuras en la mampostería. Si esto sucede, se dice comúnmente que el hogar ``sopla'', disminuyendo su rendimiento. Si es negativa, se produce el efecto contrario, absorbiendo aire en mayor cantidad que la necesaria para la combustión, provocando una disminución en la temperatura de la llama. Por tal motivo, en la práctica se debe arbitrar las medidas necesarias para tender a una presión equilibrada.

\subsection{Pérdida de carga}

En el sistema de circulación de los humos, que se origina en el hogar y llega hasta la entrada o boca de la chimenea, los gases encuentran diferentes tipos de resistencia que obstaculizan su desplazamiento. El efecto del tiro debe ser suficiente para superar estas resistencias y cumplir con su propósito.\\

Para determinar estas pérdidas se han desarrollado varios métodos de cálculo que requieren un gran número de operaciones, utilizando factores que si bien pueden ser evaluados, son difíciles de determinar. Además, tienen el inconveniente de que están referidos a sistemas que queman combustibles sólidos. Por tal motivo, actualmente es más común determinar la pérdida de carga en función de  \emph{balances de energía}.\\


Para llevar a cabo este procedimiento, se hace uso del principio de Bernoulli considerando tres puntos claves: (1) el punto inicial del sistema, el hogar, donde se generan los humos; (2) el punto de entrada de la chimenea; y (3) el punto de salida de la chimenea. Se obtienen así las siguientes dos ecuaciones:

\subsubsection{Carga motriz}

\subsubsection{Carga resistente}