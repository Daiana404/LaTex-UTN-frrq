\section{Combustión para la generación de vapor}

\subsection{Definición de combustión}
La combustión es una reacción química que se produce entre el oxígeno y un material oxidable, que va acompañada de desprendimiento de energía y habitualmente se manifiesta por incandescencia o llama.

\subsection{Temperatura de ignición}


La temperatura de autoignición o de inflamación es la temperatura mínima a la que un combustible a presión atmosférica estándar, en presencia de oxígeno (aire), puede encenderse y quemarse sin necesitar una fuente externa de calor. Es importante mencionar que esta temperatura puede variar dependiendo del tipo de combustible, ya que diferentes sustancias tienen diferentes puntos de autoignición.

\subsection{Comburente y Combustible}

\subsubsection{Comburente}
Un \emph{comburente} es una sustancia que proporciona el oxígeno necesario para que ocurra la combustión. Actúa como el agente oxidante en una reacción de combustión al suministrar el oxígeno necesario para que los combustibles ardan.


Para cálculos estequiométrico en la materia de \materia\ consideramos al aire (aire técnico) compuesto por un 79\% de nitrógeno ($N$) y un 21\% de oxígeno ($O_2$)

\subsubsection{Combustible}
El \emph{combustible} es una sustancia que tiene la capacidad de arder o quemarse. Durante la combustión, el combustible reacciona con un comburente (como el oxígeno) liberando energía en forma de calor, luz u otros productos de reacción. Los combustibles pueden ser sólidos (como la madera o el carbón), líquidos (como la gasolina o el petróleo) o gaseosos (como el gas natural).


