\hyphenation{cal-de-ras}
\section{Combustibles para calderas}
\subsection{Combustibles}

\begin{preguntas}
	. ¿Qué es un combustible?
\end{preguntas}

Un combustible es cualquier sustancia que, por un proceso químico de oxidación exotérmica\footnote{La palabra ``exotérmica'' se refiere a una reacción que libera energía térmica hacia el entorno.} o por un proceso físico de fisión o fusión, suministra energía en forma de calor.\\

Aquellas sustancias que cumplen con el primer proceso, se las denomina \textbf{combustibles industriales}. Liberan energía en forma de calor sin generar gases nocivos para el ser humano.
Esta distinción es importante, considerando que existen materiales que al quemarse producen gases tóxicos y corrosivos, como el azufre.\\

Por otro lado, el combustible que cumple con el segundo proceso es denominado como \textbf{combustible nuclear}. La combustión de estas sustancias liberan partículas radiactivas perjudiciales para la salud, lo cual implica que su uso debe llevarse a cabo en condiciones estrictas de aislamiento.

\subsection{Combustión}

La combustión es una reacción química de oxidación exotérmica donde la \textsl{energía liberada} se da en forma de \textsl{calor}. El agente oxidante ocomburente en dicha reacción es el oxígeno del aire.\\

A altas temperaturas el combustible reacciona con el aire para generar luz y gases de combustión:

\begin{equation}
	\textsl{Combustible} + \textsl{Aire} \rightarrow \textsl{Radiación} + \textsl{Gases de combustión}
\end{equation}


El combustible puede ser sólido, líquido o gaseoso, y los gases de combustión transportarán la energía en forma de calor.\\

Al ser el proceso de combustión muy complejo y con vista en aplicaciones prácticas, el estudio que se realiza en la materia \materia\ considera factores estáticos, es decir, la determinación de los estados finales e iniciales del proceso de combustión. Los resultados obtenidos a través de esta simplificación son suficientes para resolver los problemas.

\subsubsection{Tipos de combustión}

A continuación se muestran los tipos de combustión que pueden tener lugar en la reacción química.

\begin{itemize}
	\item \textbf{Combustión completa:} tiene lugar cuando \textsl{todo} el combustible se quema en presencia de suficiente oxígeno.
	\item \textbf{Combustión perfecta:} ocurre en condiciones ideales, es decir, en un entorno \textsl{libre de impurezas} y con una cantidad ilimitada de oxígeno disponible.
	\item \textbf{Combustión incompleta:} ocurre cuando un combustible se quema en presencia de una cantidad insuficiente de oxígeno, es decir, no todos los átomos del combustible se combinan con oxígeno para formar dióxido de carbono y agua, por lo que \textsl{producen subproductos} adicionales, como monóxido de carbono, alquitrán y otros hidrocarburos.
	\item \textbf{Combustión imperfecta:} se refiere a una combustión incompleta que ocurre cuando un combustible se quema en presencia de una cantidad limitada de oxígeno.
\end{itemize}


La principal diferencia entre la combustión completa y perfecta es que la última se produce bajo condiciones ideales y no produce subproductos, mientras que la combustión completa simplemente implica que todo el combustible se quema en presencia de suficiente oxígeno, sin tener en cuenta las impurezas y otros factores que pueden afectar la reacción.


Y la principal diferencia entre la combustión incompleta e imperfecta es la cantidad de oxígeno presente en la reacción. Además, la combustión incompleta es más peligrosa que la combustión imperfecta debido a la mayor cantidad de subproductos tóxicos que produce.

\subsubsection{Cálculo de aire mínimo o teórico}

\subsection{Poder calorífico}
El poder calorífico es la cantidad de energía liberada como calor en un proceso de combustión.

Se debe utilizar aquel combustible con mayor poder calorífico, aunque en la práctica no ocurre debido a la disponibilidad del combustible y su costo. También, otra limitación es cuando se tiene en cuenta la contaminación que pudiera llegar a producir a causa del humo de combustión.
	

\subsubsection{Poder Calorífico Superior}

\subsubsection{Poder Calorífico Inferior}