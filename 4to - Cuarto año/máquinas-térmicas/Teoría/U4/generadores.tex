\hyphenation{cal-de-ras}
\section{Generadores de vapor}

\subsection{Generalidades}

	IRAM-IAP-N°25/5 - ``\textit{Generadores de vapor y calderas de agua caliente}'' define como \textit{generador de vapor} al conjunto constituido por la caldera de vapor con alguno o todos de los siguientes aparatos intercambiadores de calor:
	\begin{itemize}
		\item \textbf{Sobrecalentador:} se utiliza en las calderas para aumentar la temperatura del vapor generado por encima de su punto de saturación. Esto se logra al transferir calor adicional al vapor después de que ha sido generado en la caldera. El sobrecalentador ayuda a mejorar la eficiencia y el rendimiento del sistema al proporcionar vapor seco y sobrecalentado.
		
		\item \textbf{Desobrecalentador:} se utiliza para reducir la temperatura del vapor sobrecalentado. Su función principal es enfriar el vapor y asegurarse de que no exceda los límites de temperatura deseados antes de entrar en ciertos componentes o procesos posteriores.
		
		\item \textbf{Recalentador:} se utiliza para aumentar la temperatura del vapor. Sin embargo, a diferencia del sobrecalentador, coloca en una ubicación específica dentro del ciclo de vapor para volver a calentar el vapor después de que ha pasado por una etapa de expansión. Esto ayuda a aumentar la eficiencia térmica del sistema.
		
		\item \textbf{Economizador:} se utiliza para aprovechar el calor residual de los gases de escape de la caldera y transferirlo al agua de alimentación de la caldera. Al precalentar el agua de alimentación antes de entrar en la caldera, se logra un ahorro de energía al reducir la cantidad de calor requerido para generar vapor.
		
		\item \textbf{Calentador de aire:} se utiliza para calentar el aire que se suministra a un proceso o sistema. El aire frío que ingresa al calentador de aire se calienta al hacerlo pasar a través de tubos o conductos donde se transfiere calor desde una fuente de calor, como el vapor o los gases de escape. El aire caliente resultante se utiliza en diversas aplicaciones industriales o comerciales que requieren aire caliente.
	\end{itemize}

\subsection{Calderas}

La caldera, frente a los demás dispositivos, es el elemento principal e indispensable de un generador de vapor ya que los otros pueden no existir. La instalación o no de el resto de intercambiadores de vapor influirá en la calidad de vapor que se desee obtener y la conveniencia de aprovechar la energía calorífica que aún contienen los gases de combustión luego de su paso por la caldera.

\subsubsection{Clasificación}

En general, y a efectos de la materia \materia\ las calderas se clasifican según cómo circula el agua y el humo por las mismas. De esta manera, se distinguen dos tipos: calderas humotubulares o pirotubulares, y calderas acuotubulares.

\begin{figure}[!h]
	\centering
	\begin{subfigure}[b]{.4\linewidth}
		\includegraphics[width = \linewidth]{humotubular}
		\caption{Caldera humotubular}
		\label{fig:caldera-humo}
	\end{subfigure}
	\begin{subfigure}[b]{.4\linewidth}
		\includegraphics[width = \linewidth]{acuotubular}
		\caption{Caldera acuotubular}
		\label{fig:caldera-acuo}
	\end{subfigure}
	\caption{Tipos de calderas}
\end{figure}

\minisection{Calderas humotubulares}

Las calderas humotubulares (figura \ref{fig:caldera-humo}) son aquellas en la cual \textbf{los gases de combustión o humos circulan por el interior de tubos} que se encuentran sumergidos en el agua de la caldera (agua a vaporizar). Se definen en las normas como aquellas que cuentan con un ``cuerpo de la caldera'', el cual es una envolvente cilíndrica cerrada donde se produce la circulación del agua y se mantiene su nivel en el caso de calderas de vapor, o donde se produce el calentamiento del agua en el caso de calderas de agua caliente. Dentro de este ``cuerpo'', se disponen haces de tubos con diámetros relativamente pequeños por los cuales circulan los gases de combustión o humos.\\


Este tipo de calderas ofrece una \textbf{distribución más uniforme del calor en la masa de agua}, lo que resulta en una generación de vapor y un rendimiento más uniforme en comparación con las calderas acuotubulares. Además, su puesta en marcha es más rápida. Su posición puede ser horizontal, con hogar exterior, o vertical, con hogar interior.\\



En la actualidad son más utilizadas las calderas humotubulares de hogar interior, quemando un combustible industrial líquido o gaseoso permitiendo una producción de vapor de hasta $15\frac{tn}{h}$. Las humotubulares verticales se utilizan para bajas producciones de vapor y presiones de hasta $15\frac{kg}{cm^2}$


\minisection{Calderas acuotubulares}

En las calderas acuotubulares (figura \ref{fig:caldera-acuo})  \textbf{el agua y el vapor producido circulan por el interior de los tubos} cuya superficie exterior está en contacto con los gases calientes de la combustión.\\

Los elementos básicos son el \emph{domo} y el
\emph{haz de tubos}. El domo es un cuerpo cilíndrico mayor hacia
el cual convergen los tubos de la caldera y sirven como colector de vapor y/o agua. Están montados normalmente en la parte superior de la caldera. El haz de
tubos lo constituyen el conjunto de tubos por donde circula el agua por convección y se disponen por debajo del domo.\\

En las calderas antiguas, muchas de ellas todavía en uso, el haz de tubos está unido a un colector denominado \emph{placa de tubos}, el cual sirve de sostén a los tubos en sus extremos en las calderas. Esta placa forma parte del cabezal que se une al domo.\\

Los tubos están montados con una cierta inclinación hacia la parte posterior de la caldera para que el agua caliente y el vapor que se van generando circulen hacia adelante, y además permite el depósito de los ``barros'' en el colector de lodos montado en la parte inferior.\\

La necesidad de un gran producción de vapor, principalmente para la generación de energía eléctrica, sin aumentar proporcionalmente las dimensiones del generador, provocó un gran
desarrollo de los mismos, llevando a la construcción de calderas con más de un domo y haces de tubos de pequeños diámetros, cuya disposición en el conjunto depende del fabricante. El utilizar tubos de pequeños diámetros es a efectos de aumentar la superficie de intercambio
calórico.\\

Debido a que se puede producir un deterioro, las paredes refractarias del hogar se cubren tubos por cuyo interior circula agua. Las mismas, además de su carácter protector aumentan la superficie de intercambio de la caldera, y consecuentemente un mejor aprovechamiento del calor de radiación.\\


Las calderas acuotubulares pueden clasificarse en dos grandes grupos:
\begin{enumerate}[label=\alph*)]
	\item \textbf{De tubos rectos:} los mismos pueden estar montados en posición vertical u horizontal, y en algunos casos con pequeña desviación para favorecer
la circulación del agua líquida y acumulación de los lodos.


	Entre las ventajas se encuentran:
	\begin{itemize}
		\item Al ser los tubos de igual diámetro, el stock para futuros cambios es reducido.
		\item La limpieza interior de los tubos es sencilla.
		\item El cambio de un tubo no exige mover otros.
		\item Su diseño puede realizarse para poca altura.
	\end{itemize}
	Como desventajas se tiene:
	\begin{itemize}
		\item Dificultad en la construcción de los cabezales colectores.
		\item Gran número de agujeros a realizar en la placa de tubos.
		\item La limpieza de tubos exige retirar gran cantidad de elementos de cierre.
		\item La dilatación de los tubos
	\end{itemize}
	\item \textbf{De tubos curvados:} este tipo de calderas pueden tener hasta cinco domos, siendo las más comunes la de dos domos. La superficie de intercambio de calor de mayor magnitud lo forman la superficie de los tubos que vinculan entre sí a los domos. Los tubos se encuentran dispuestos en dos o más haces, separados por pantallas horizontales y/o verticales, o una combinación de ambas para una mejor circulación de los gases de combustión.
	
	Entre las ventajas, se pueden mencionar las siguientes:
	\begin{itemize}
		\item Elasticidad para el diseño.
		\item La curvatura permite la dilatación de los tubos sin afectar su unión al domo.
		\item Existe una mejor circulación del agua y gases, lo que permite mayor producción de vapor.
	\end{itemize}

	 Como desventajas, tenemos:
	 \begin{itemize}
	 	\item Al contar con distintos tipos de tubos (diámetro, largo, etc), se debe tener un stock bastante amplio.
	 	\item La limpieza interior del tubo se dificulta principalmente en las curvas.
	 	\item El reemplazo de tubos exige, en algunos casos, el retiro de otros.
	 	\item 
	 \end{itemize}
\end{enumerate}

\subsubsection{Accesorios de una caldera}

Los accesorios de una caldera son reglamentarios y tienen
por finalidad que la cantidad de agua en el interior de la caldera sea la conveniente y que la presión interna debido al vapor generado no sobrepase el límite establecido (presión de timbre).

\begin{enumerate}[label=\alph*)]
	\item \textbf{Nivel de agua:} se colocan indicadores de nivel de agua visuales y sonoros. El indicador visual es más preciso ya que el nivel de agua se observa a simple vista. El sonoro funciona generalmente por acción de un flotador, que al bajar el nivel del agua regulado deja escapar vapor que produce un sonido característico.
	\item \textbf{Sobrepresión:} los accesorios que evitan que la presión de vapor no sobrepase la de trabajo fijada son dos válvulas de seguridad (a contrapeso y a resorte) y un manómetro de cuadrante con números de suficiente dimensión para ser observados sin dificultad.
	
	Las válvulas permiten la liberación de vapor en casos de sobrepresión, mientras que el manómetro indica el valor de dicha presión.
\end{enumerate}


\subsubsection{Zonas de una caldera}

\minisection{Zona de liberación de calor}

En esta zona, llamado hogar, el calor se transfiere al agua principalmente por radiación. Es una zona crítica desde el punto de vista de resistencia de los materiales. 

\minisection{Zona de tubos}

Es la zona donde los gases, productos de la combustión, transfieren calor al agua principalmente por convección a medida que circulan por su circuito.

\subsubsection{Partes de una caldera}

\begin{enumerate}
	\item \textbf{Hogar} 
	
	El hogar es el espacio dentro de la caldera donde se realiza la combustión del combustible y la liberación de calor, generalmente ubicada en la parte inferior de la caldera. El mismo puede estar situado de dos formas:
	\begin{itemize}
		\item En el interior, donde se encuentra dentro de un recipiente metálico rodeado de paredes refrigeradas con agua.
		\item En el exterior, donde está construido fuera del recipiente metálico y puede estar parcialmente rodeado o sin paredes refrigeradas.
	\end{itemize}
	
	Se puede clasificar según el tipo de combustible, para sólidos o gaseosos; o el tipo constructivo, de tubo liso o corrugado.
	\item \textbf{Puerta de hogar}
	
	Es una pieza metálica robusta con abisagrada que puede tener varias funciones según el tipo de caldera. Si se trata de un combustible sólido, es utilizada para la \textit{carga manual del combustible} y la \textit{inspección de la llama}. Si se quema combustible líquido o gaseoso, esta puerta se reemplaza por un quemador.
	
	\item \textbf{Emparrillado}
	
	Es una estructura metálica de forma enrejada y sirve de apoyo para el combustible sólido depositado en el hogar. Permite el ingreso de aire primario para dr origen a la combustión.
	
	
	Según su diseño puede ser \textit{parrilla fija}, seca o húmeda; o \textit{parrilla móvil}, como transportadores, reciprocantes o basculantes.\\
	
	Un buen emparrillado debe cumplir con las siguientes condiciones:
	\begin{itemize}
		\item Permitir el paso del aire primario.
		\item Permitir que las cenizas caigan hacia el cenicero.
		\item Que se limpie con facilidad y rapidez.
		\item Impedir la acumulación de escoria.
		\item Ser construidos con materiales de buena calidad para que no se quemen, deformen y perduren en el tiempo.
	\end{itemize}

	\item \textbf{Cenicero}
	
	Es el espacio que queda bajo la parrilla y sirve para recoger las cenizas que caen. Los residuos acumulados deben ser retirados periódicamente para no obstaculizar el paso del aire primario.
	
	
	La extracción de las cenizas puede ser manual o mecanizada.
	
	\item \textbf{Puerta del cenicero}
	
	De igual características constructivas que la puerta del hogar, esta se utiliza para realizar funciones de limpieza del cenicero. Además, con esta puerta se puede regular la entrada de aire primario si la caldera no tiene tiraje forzado.
	
	\item \textbf{Altar}
	
	Es un pequeño muro de ladrillo refractario, debiendo pasar una altura aproximada de 30 cm, ubicado en el extremo opuesto a la puerta del fogón y al final de la parilla. Las funciones del altar son:
	\begin{itemize}
		\item Impedir que caigan a la parrilla partículas de combustible sin quemar.
		\item Ofrecer resistencia a la llama y gases para que estos se distribuyan a lo ancho del hogar pudiendo entregar la mayor parte de su energía al agua.
		\item En algunos casos sirve para ubicar boquillas para el ingreso de aire secundario.
	\end{itemize}

	\item \textbf{Mampostería}
	
	La mampostería es la construcción de ladrillos refractarios y comunes que tienen por objetivo \textit{cubrir la caldera} para minimizar las pérdidas de calor, y \textit{guiar los gases} y humos calientes en su recorrido.
	
	
	Para mejorar la aislación, a veces se construyen paredes con espacios huecos.
	
	
	En la actualidad, en algunos tipos de calderas se ha eliminado totalmente la mampostería de ladrillo, colocándose aislación térmica con materiales tales como lana de vidrio de alta densidad recubierta con chapas metálicas galvanizadas o inoxidables.
	
	\item \textbf{Conductos de humo}
	
	Son los espacios por los cuales circulan los humos y gases calientes de la combustión.
	
	\item \textbf{Caja de humo}
	
	Corresponde al espacio de la caldera en el cual se juntan los humos después de haber entregado su energía y antes de salir por la chimenea.
	
	\item \textbf{Chimenea}
	 
	Es un tubo que puede ser construido de mampostería o de chapa de acero. Se utiliza para conducir los gases de combustión a la atmósfera, ademas tiene la función de producir un tiraje natural.
	
	\item \textbf{Reguladores de tiro}
	
	Consiste en una compuerta metálica instalada en el conducto de humo que comunica con la chimenea o bien en la chimenea misma. Tiene por objetivo regular la cantidad de aire necesario para la combustión en función de la demanda de vapor, a su vez, ayuda a regular el paso de los gases de combustión a la salida.
	
	\item \textbf{Cilindro o tambor}
	
	El tambor de la caldera está diseñado para almacenar agua y vapor. En las calderas de tipo acuotubular, se encuentra conectado a una serie de tubos en los cuales circulan los gases calientes provenientes de la combustión. Estos tubos se encuentran sumergidos en el agua del tambor, lo que permite transferir el calor de los gases al agua y generar vapor.
	En el tambor se pueden diferenciar dos zonas principales: la cámara de agua y la cámara de vapor.
	
	La cámara de agua en el tambor de una caldera es un espacio específico dentro del tambor donde se ubica el nivel de agua y se lleva a cabo la separación entre el vapor y el agua líquida.
	
\end{enumerate}   
