\documentclass[11pt,a4paper,twocolumn]{article}
\usepackage[utf8]{inputenc}
\usepackage[T1]{fontenc}
\usepackage{amsmath}
\usepackage{amssymb}
\usepackage{amsfonts}
\usepackage{graphicx}
\usepackage[spanish]{babel}
\usepackage{tcolorbox,booktabs,fourier,tabularx,wrapfig,multicol,caption, subcaption,tikz,fancyhdr, tcolorbox, esint}
\usepackage[left=2cm,right=2cm,top=2cm,bottom=2cm]{geometry}
\tcbuselibrary{breakable}

\graphicspath{{../imagenes/}}


%Comando título de cada unidad
\newcommand{\unidad}[2]{\begin{center}
		\fontsize{10}{10}\selectfont\color{gray!50!black}\scshape Unidad #1 \\
		\fontsize{14}{14}\selectfont \scshape #2
	\end{center} \vspace{-.5cm}}



%\author{}
%\title{\includegraphics[width=.12\textwidth]{utn} \\ \textsc{Termodinámica} \\ \textsl{Hoja de fórmulas} \\ }
%\date{}

%Encabezado y pie de página


\fancyhead[R]{\textsc{Termodinámica Técnica}}
\fancyhead[L]{\includegraphics[width=.1\textwidth]{utncom}}
\fancyhead[C]{ Hoja de fórmulas}
\fancyfoot[L]{Franco Guardiani, Valentin Franzoi, Daiana Polo}
\fancyfoot[R]{\thepage}
\fancyfoot[C]{}

\begin{document}
	

	
\pagestyle{fancy}
\vspace{1cm}
	\section*{Nomenclatura} \vspace{-.3cm}
	\begin{tabular}{r l}
		$P$ [Pa] & Presión\\
		$V$ [m$^{3}$] & Volumen \\
		$m$ [kg] & Masa \\
		$\delta$ [kg/m$^{3}$] & Densidad \\
		& $\delta_{H_{2}O}=997$[kg/m$^{3}$] \\
		$l$ [m] & Altura \\ 
		$g$ [m/s$^{2}$] & Aceleración de la gravedad \\
		$\dot{m}$ [kg/min] & Caudal másico \\
		$\dot{Q}$ [m$^3$/min]& Caudal volumétrico\\
		$T$ [K] & Temperatura \\
		$R$ [kJ/kg$\cdot$K] & Constante del gas \\
		$\overline{R}$ [kJ/kg$\cdot$K] & Constante universal de los gases \\
		& $\overline{R}=8.31434 [kJ/kmol\cdot K]$ \\
		$n$ [moles] & Número de moles \\
		$M$ [kg/kmol] & Peso molecular \\ 
		$\overline{v}$[m$^3$/kmol] & Volumen específico molar \\
		$t$ [s] & Tiempo \\
		$Q$ [J] & Calor \\
		$u,U$ [J] & Energía interna\\
		$W$ [J] & Trabajo \\
		$W_{c}$ [J] & Trabajo de circulación\\
		$h, H$ [J]& Entalpía \\
		$c_{v}$ & Calor específico a $v=cte$ \\
		$c_{p}$ & Calor específico a $p=cte$ \\
		$c_{n}$ & Calor específico medio del gas \\
		$k$ & Razón de calores específicos \\
		$\gamma$ & Coeficiente adiabático \\
		$n$ & Coeficiente politrópico \\
		$\Delta$ & Estado 2 $-$ Estado 1 \\
		$x_{i}$ & Fracción molar \\
		$g_{i}$ & Composición gravimétrica \\
		$P_{c}, T_{c}, V_{c}$ & Valores críticos \\
		$z$ [m] & Altura \\
		$\omega$[m/s$^{2}$] & Velocidad del gas\\

		
	\end{tabular}
\newpage

%%%%%%%%%%%%%%%%%%%%%%%%%%%%%%%%%%%%%%%%%%%%%%%%%%%%%%%%%%%%%%%%%%%%%%%%%%%%%%%%%%%%%%%%%%%%%%%%%%%%%%%%%%%%%%%%%%%%%%%%%%%%%%%%%%%%%%%%%%%%%%%%
\begin{center}
\color{gray!50!black}	\fontsize{14}{14}\selectfont \scshape Fórmulas más utilizadas
\end{center}\vspace{-.5cm}
\begin{tcolorbox}[colback=white!90!gray, colframe=black!80!gray]
	\begin{tabular}{r l}
\vspace{.1cm}		Ec. General & $pV=mRT$ / $pv=RT$\\

\vspace{.1cm}		1er Principio & $Q=\Delta h+\Delta E_{p}+\Delta E_{c}+W_{c}$\\ 
\vspace{.1cm}		2do Principio & $Q_{1}=W+Q_{2}$\\

\vspace{.1cm}		Calor & $\delta Q=c\cdot  dT$ \\
\vspace{.1cm}		Energía interna & $du=c_{v} \cdot dT$ \\
\vspace{.1cm}		Entalpía & $ dh = c_{p} \cdot dT $ \\
\vspace{.1cm}		Relación Mayer & $R=c_{p}-c_{v}$\\
\vspace{.1cm}		Constante del gas & $\overline{R}= R \cdot M$\\ 
\vspace{.1cm}		Trabajo SC & $\delta W = p \cdot dv$ \\
\vspace{.1cm}		Trabajo SA rp & $\delta W_c = -v \cdot dp$\\

\vspace{.1cm}		Rendimiento & $\eta = \frac{obtenido}{demandado}$ \\
\vspace{.1cm}		Ciclo de Carnot & $\eta = 1-\dfrac{Q_2}{Q_1}=1-\dfrac{T_2}{T_1}$ \\
\vspace{.1cm}		Politrópica & $pv^{n}=cte$ $$\\
\vspace{.1cm}		Potencia & $ N=W \dot{m}=Q \dot{m} $ \\

	
	\end{tabular}
\end{tcolorbox}

%%%%%%%%%%%%%%%%%%%%%%%%%%%%%%%%%%%%%%%%%%%%%%%%%%%%%%%%%%%%%%%%%%%%%%%%%%%%%%%%%%%%%%%%%%%%%%%%%%%%%%%%%%%%%%%%%%%%%%%%%%%%%%%%%%%%%%%%%%%%%%%%

\unidad{1}{Conceptos Fundamentales}
\vspace{0.1cm}
	\begin{tcolorbox}[colback=white!97!brown, colframe=brown!15!gray]
		\begin{tabular}{r l}
			
			\vspace{.1cm}	Presión hidrostática 		& $P_{h}=\delta g l$ \\
			\vspace{.1cm}	 $P_{abs}=P_{atm}+P_{man}$	 & $P_{vacío} = P_{atm} - P_{abs}$ \\ 
			\vspace{.1cm}	Densidad específica			 & $\delta_{e} = \dfrac{\delta}{\delta_{H_{2}O}}=\dfrac{m}{V \delta_{H_{2}O}}$ \\
			\vspace{.1cm} Volumen específico 			& $v_{e}/v=\dfrac{V}{m} \hspace{.2cm} \overline{v}=\dfrac{V}{m}M$ \\

			\vspace{.1cm} Potencia 						& $N=W\cdot \dot{m}$\\
			\vspace{.1cm}Temperatura 					& \\ \multicolumn{2}{c}{$\dfrac{^{\circ}C}{100}=\dfrac{^{\circ}F-32}{180}=\dfrac{K-273}{100}$} \\	
	\end{tabular}
	\end{tcolorbox}

%%%%%%%%%%%%%%%%%%%%%%%%%%%%%%%%%%%%%%%%%%%%%%%%%%%%%%%%%%%%%%%%%%%%%%%%%%%%%%%%%%%%%%%%%%%%%%%%%%%%%%%%%%%%%%%%%%%%%%%%%%%%%%%%%%%%%%%%%%%%%%%%

	\unidad{2}{Gases Ideales}
	\begin{tcolorbox}[colback=white!97!brown, colframe=brown!15!gray]
		
		\begin{tabular}{r l}
		\vspace{.1cm}	\textbf{Ecuación general} & $PV=mRT$ \\
		
		
		\vspace{.1cm}	Ley Boyle-Mariotte 	& $PV=cte$  ($T=cte$)\\
		\vspace{.1cm}	Ley de Charles			& $\dfrac{V}{T}=cte$  ($P=cte$)\\
		
		\vspace{.1cm}	Ley de Joule& $u= c_{v} T$ \\
		\vspace{.1cm}	Entalpía 	& $h=u+pv$ \\
		\vspace{.1cm}			 	& $h=c_{p}T$ \\
		
		\vspace{.1cm}	$PV = n \overline{R} T$ & $m=nM$ \hspace{.2cm} $\overline{R} = R M$\\
		\vspace{.1cm}	 Trabajo			 & $\delta W= p \cdot dv  \hspace{.2cm} [J/kg] $ \\
		\vspace{.1cm}	 Calor 					& $\delta Q=c \cdot dT \hspace{.35cm} [J/kg] $  \\
	
	\vspace{.2cm}		Relación \textbf{\textsc{importante}} &  $\dfrac{pv}{T}=R=cte$ \\	
		\end{tabular}\\


	\end{tcolorbox}

\newpage

%%%%%%%%%%%%%%%%%%%%%%%%%%%%%%%%%%%%%%%%%%%%%%%%%%%%%%%%%%%%%%%%%%%%%%%%%%%%%%%%%%%%%%%%%%%%%%%%%%%%%%%%%%%%%%%%%%%%%%%%%%%%%%%%%%%%%%%%%%%%%%%%


		\unidad{3}{Gases Reales}
	
	\begin{tcolorbox}[colback=white!97!brown, colframe=brown!15!gray]
		
		
		
		\begin{tabular}{r l}
			\vspace{.1cm}	Ecuación de Mayer 		& $R=c_{p}-c_{v}$ \\
			\vspace{.2cm}			Razón de calores 			& $k = \dfrac{c_{p}}{c_{v}}>1$ \\
			\multicolumn{2}{l}{\textbf{Mezcla de gases}} \\
			\vspace{.1cm}	Ley de Dalton 			& $P_{T}=\displaystyle\sum P_{i}$ \\
%			\multicolumn{2}{c}{\small $(P_{1}+P_{2})V= n_{T}\overline{R}.T$ }\\
			\vspace{.1cm}Ley de Amagat 				& $V_{T}=\displaystyle\sum V_{i}$ \\
%			\multicolumn{2}{c}{\vspace{.2cm} \small $P(V_{1}+V_{2})= n_{T}\overline{R}.T$}\\
			
			\vspace{.1cm}\textbf{Fracción molar} & $x_{i}=\dfrac{n_{i}}{n_{T}}=\dfrac{V_{i}}{V_{T}}=\dfrac{P_{i}}{P_{T}}$\\
			\vspace{.1cm} Compos. gravimétrica & $g_{i}=\dfrac{m_{i}}{m_{T}}$\\
			                                                                 
			\vspace{.1cm} $u=\sum g_{i}u_{i}$ 		&$c_{v}= \sum g_{i}c_{vi}$\\
			\vspace{.2cm} $h=\sum g_{i}h_{i}$ 		&$c_{p}=\sum g_{i}c_{pi}$\\
			\vspace{.2cm} Peso molecular & $M_T=\dfrac{m_T}{n_T}$ \\
			
			\multicolumn{2}{l}{\vspace{.1cm}	Van der Waals } \\
			\multicolumn{2}{c}{$\left( p+\dfrac{a}{(\overline{v})^{2}} \right)=\dfrac{\overline{R}T}{(\overline{v}-b)}$} \\
			\multicolumn{2}{c}{\vspace{.2cm} $a=3P_{c} V_{c}^{2} \hspace{.2cm} b=\dfrac{V_{c}}{3} \hspace{.2cm} R=\dfrac{8}{3}\dfrac{P_{c}V_{c}}{T_{c}}$ } \\
			
			\multicolumn{2}{l}{\vspace{.1cm} Beattie-Bridgeman} \\
			\multicolumn{2}{c}{\vspace{.1cm}$ \left( p+\dfrac{A}{(\overline{v})^2} \right) \dfrac{(\overline{v})^2}{(\overline{v}+B)}=\overline{R}T(1-e) $} \\
			\multicolumn{2}{c}{\vspace{.1cm}$ A=A_{0} \left( 1- \dfrac{a}{\overline{v}} \right) \hspace{.2cm}  B=B_{0} \left( 1- \dfrac{b}{\overline{v}} \right) \hspace{.2cm} e=\dfrac{c}{\overline{v}T^{3}} $} \\
	\vspace{.1cm}		Propiedades reducidas & \\
			\multicolumn{2}{c}{\vspace{.2cm} $ p_{r}=\dfrac{p}{p_{c}} T_{r}=\dfrac{T}{T_{c}} v_{r}=\dfrac{v}{v_{c}}$} \\
			\vspace{.1cm} Gou Yen Sou & $PV=zn\overline{R}T$ \\
		\end{tabular}
	\end{tcolorbox}

\newpage
%%%%%%%%%%%%%%%%%%%%%%%%%%%%%%%%%%%%%%%%%%%%%%%%%%%%%%%%%%%%%%%%%%%%%%%%%%%%%%%%%%%%%%%%%%%%%%%%%%%%%%%%%%%%%%%%%%%%%%%%%%%%%%%%%%%%%%%%%%%%%%%%


	\unidad{4}{Transformaciones en Gases}
	
	\begin{tcolorbox}[colback=white!97!brown, colframe=brown!15!gray, breakable]
		
	
		\begin{tabular}{r l}
		
			\multicolumn{2}{l}{En \textbf{sistemas cerrados}} \\

\vspace{.1cm}			{\textcolor{brown!50!black}{\textbf{Primer principio}}}& $Q = \Delta U + W$\\
\vspace{.1cm}			W(+): $\Box \rightarrow$& Q(+): $\Box \leftarrow$\\
\vspace{.1cm}			W(-): $\Box \leftarrow$  &   Q(-): $\Box \rightarrow$ \\

		

	\multicolumn{2}{l}{\vspace{.1cm} En \textbf{sistemas abiertos}} \\
	\vspace{.1cm}		{\textcolor{brown!50!black}{\textbf{Primer principio}}} & $Q=\Delta E+W_{T}$\\
	\multicolumn{2}{c}{$Q=\Delta h + \Delta E_{p} +\Delta E_{c} + W_{c}$} \\
	%	\vspace{.1cm}						& $W_{T}=-W_{c}+L_{2}-L_{1}$\\ 
	\vspace{.1cm}		Energía potencial & $E_p = gz \left[\frac{J}{kg}\right]$ \\
	\vspace{.1cm}		Energía cinética & $E_c = \frac{1}{2}\omega^{2} \left[\frac{J}{kg}\right]$ \\
	\vspace{.1cm}		Trabajo circulante &$W_{c}=\displaystyle\int_{1}^{2} -v \cdot dp$\\
	%	\vspace{.1cm}	Si $E_{c}=E_{p}=0$	& $Q=\Delta H +W_{c}$\\
	\vspace{.2cm}	Si no hay datos para $E_{c/p}$ & $\Rightarrow \hspace{.1cm} E_{c/p}=0$ \\
	
	\vspace{.1cm} Caudal volumétrico 			& $\dot{Q}=A\cdot \omega [m^{3}/s] $ \\
	\vspace{.1cm} Caudal másico					& $\dot{m}= \dfrac{\dot{Q}}{v_{e}} \ [kg/s] $ \\

\end{tabular}

\vspace{.1cm}
  \textbf{Transformaciones} 
\vspace{.2cm}	

	\resizebox{7.7cm}{!}{\begin{tabular}{r l l l }
	\vspace{.1cm}	Isocórica 	& $pv^{\infty}=cte$ 	& $ Q = \Delta u =c_{v} \Delta T$ 				& $W=0$\\
	\vspace{.1cm}	Isobárico  	& $pv^{0}=cte$ 			& $ Q = \Delta h = c_{p} \Delta T$ 				& $W=p(v_{2}-v_{1})$ \\ 
	\vspace{.1cm}	Isotérmico	&$pv=cte$				& $Q= p_{1}v_{1} \ln{\dfrac{v_{2}}{v_{1}}}$ 		& $W=p_{1}v_{1} \ln{\dfrac{v_2}{v_1}}$\\ 
	\vspace{.1cm}	Adiabática  &$pv^{\gamma}=cte$ 		& $Q=0$											& $W=\dfrac{p_{1}v_{1}-p_{2}v_{2}}{\gamma -1}$\\
	\vspace{.1cm}	Politrópica & $pv^{n}=cte$ 			& $Q=\left(\dfrac{\gamma-n}{\gamma-1}\right)W$	& $W=\dfrac{p_{1}v_{1}-p_{2}v_{2}}{n -1}$\\
	\end{tabular}}	
	
\vspace{.3cm}
\textsl{Para politrópicas}
\vspace{.1cm}

\hspace{-.5cm}
\resizebox{8.3cm}{!}{\begin{tabular}{c c c}
	$T v^{n-1}=cte$ & $p v^{n}=cte$ & \vspace{.2cm} $TP^{\dfrac{1-n}{n}}=cte$ \\
	$W=\dfrac{R T_{1}}{n-1}\left(1-\dfrac{T_2}{T_1}\right)$  & 
	$W=\dfrac{R T_{1}}{n-1}\left[1-\left(\dfrac{v_1}{v_2}\right)^{n-1}\right]$ & 
	$W=\dfrac{R T_{1}}{n-1}\left[1-\left(\dfrac{p_2}{p_1}\right)^{\frac{n-1}{n}}\right]$
\end{tabular}}



%	En general:
%	\begin{center}
%		$Q=h_{2}+E_{c2}+E_{p2}-h_{1}-E_{c1}-E_{p1}+W_{c}$ 
%	\end{center}
%	Relación potencia-trabajo-caudal:
%	\begin{center}
%		$N=W \dot{m}$
%	\end{center}


	\end{tcolorbox}
%------------------------------------------------------------------------------------------------------------------------
		
%	\begin{center}
%		\begin{table}[htbp]
%			
%			\centering
%			\resizebox{8.5cm}{!}{ %ACA REESCALA LA TABLA 
%				\begin{tabular}{|l|c|l|l|l|l|}
%					\hline
%					\multicolumn{1}{|c|}{\textbf{Proceso}}                          & \textbf{\begin{tabular}[c]{@{}c@{}}Índice\\ n\end{tabular}} & \multicolumn{1}{c|}{\textbf{\begin{tabular}[c]{@{}c@{}}Calor \\ agregado\end{tabular}}} & \textbf{$\int_{1}^{2} p.dv$} & \multicolumn{1}{c|}{\textbf{\begin{tabular}[c]{@{}c@{}}Relaciones\\ p,v,T\end{tabular}}} & \multicolumn{1}{c|}{\textbf{\begin{tabular}[c]{@{}c@{}}Calor específico\\ c\end{tabular}}} \\ \hline
%					\begin{tabular}[c]{@{}l@{}}Presión\\ Constante\end{tabular} 
%					& n=0 & $\mathit{c_{p}}.(T_{2}-T_{1})$  &    $p.(\mathit{v_{2}}-\mathit{v_{1}})$ & $\dfrac{T_{2}}{T_{1}}=\dfrac{\mathit{v_{2}}}{\mathit{v_{1}}}$ & $\mathit{c_{p}}$
%					\\ \hline
%					\begin{tabular}[c]{@{}l@{}}Volumen \\ Constante\end{tabular}    
%					& n=$\infty$& $\mathit{c_{v}}.(T_{2}-T_{1})$  &    0 & $\dfrac{T_{1}}{T_{2}}=\dfrac{\mathit{p_{1}}}{\mathit{p_{2}}}$ & $\mathit{c_{v}}$
%					\\ \hline
%					\begin{tabular}[c]{@{}l@{}}Temperatura\\ Constante\end{tabular} 
%					& n=1&$\mathit{p_{1}}.\mathit{v_{2}}.\log_{e}\dfrac{\mathit{v_{2}}}{\mathit{v_{1}}} $ &    $\mathit{p_{1}}.\mathit{v_{2}}.\log_{e}\dfrac{\mathit{v_{2}}}{\mathit{v_{1}}} $ & $ \mathit{p_{1}}.\mathit{v_{1}}=\mathit{p_{2}}.\mathit{v_{2}} $ & $\infty$
%					\\ \hline
%					\begin{tabular}[c]{@{}l@{}}Adiabático\\ reversible\end{tabular} & n=$\gamma$ &  0 & $ \dfrac{\mathit{p_{1}}.\mathit{v_{1}}-\mathit{p_{2}}.\mathit{v_{2}}}{\gamma-1} $ &
%					\begin{tabular}[c]{@{}l@{}} 
%							\vspace{0.25cm}$ \mathit{p_{1}}.\mathit{v_{1}}^{\gamma}=\mathit{p_{2}}.\mathit{v_{2}}^{\gamma} $\\ 	\vspace{0.25cm}$\dfrac{T_{2}}{T_{1}}=(\dfrac{\mathit{v_{1}}}{\mathit{v_{2}}})^{\gamma-1}$\\
%							\vspace{0.25cm}$\dfrac{T_{2}}{T_{1}}=(\dfrac{\mathit{p_{2}}}{\mathit{p_{1}}})^{\dfrac{\gamma-1}{\gamma}}$
%					\end{tabular}
%					
%					&         0   \\ 
%					\hline
%					Politrópico  & n=n &
%					\begin{tabular}[c]{@{}l@{}} 
%						\vspace{0.25cm}$ \mathit{c_{n}}.(T_{2}-T_{1}) $\\
%						\vspace{0.25cm}$=\mathit{c_{v}}.(\dfrac{\gamma-n}{1-n}).(T_{2}-T_{1})	$\\
%						\vspace{0.25cm}$=(\dfrac{\gamma-n}{1-n}).\mathcal{W}_{sin~flujo}$\\
%					\end{tabular} 
%					&$ \dfrac{\mathit{p_{1}}.\mathit{v_{1}}-\mathit{p_{2}}.\mathit{v_{2}}}{n-1} $ 
%					&\begin{tabular}[c]{@{}l@{}} 
%							\vspace{0.25cm}$\mathit{p_{1}}.\mathit{v_{1}}^{n}=\mathit{p_{2}}.\mathit{v_{2}}^{n} $\\  	\vspace{0.25cm}$\dfrac{T_{2}}{T_{1}}=(\dfrac{\mathit{v_{1}}}{\mathit{v_{2}}})^{n-1}$\\
%							\vspace{0.25cm}$\dfrac{T_{2}}{T_{1}}=(\dfrac{\mathit{p_{2}}}{\mathit{p_{1}}})^{\dfrac{n-1}{n}}	$
%					\end{tabular}
%					& $\mathit{c_{n}}=\mathit{c_{v}}.(\dfrac{\gamma-n}{1-n})$
%					\\ \hline
%				\end{tabular}
%			}
%		\end{table}
%	\end{center}
%------------------------------------------------------------------------------------------------------------------------

%%%%%%%%%%%%%%%%%%%%%%%%%%%%%%%%%%%%%%%%%%%%%%%%%%%%%%%%%%%%%%%%%%%%%%%%%%%%%%%%%%%%%%%%%%%%%%%%%%%%%%%%%%%%%%%%%%%%%%%%%%%%%%%%%%%%%%%%%%%%%%%%


%\unidad{5}{Segundo principio}
%	\begin{tcolorbox}[colback=white!97!brown, colframe=brown!15!gray]
%		\begin{center}
%			\emph{La transferencia de calor y trabajo no es equivalente, existen pérdidas.}\\
%			
%			$Q_{1}=W+Q_{2}$
%		\end{center}
%	\begin{tabular}{r l}
%\vspace{.1cm}		Rendimiento & $\eta = \dfrac{\textup{energía útil}}{\textup{energía absorbida}}$ \\
%\vspace{.1cm}		Ciclo de Carnot & $\eta = 1 - \dfrac{Q_{2}}{Q_{1}}=1-\dfrac{T_{2}}{T_{1}}$ \\
%\vspace{.1cm}		Teorema de Clausius & $\displaystyle\sum \dfrac{Q_{i}}{T_{i}} \leq 0$\\
%\vspace{.1cm}		$\displaystyle\sum \dfrac{Q_{i}}{T_{i}} = 0$ & Procesos reversibles \\
%\vspace{.1cm}		$\displaystyle\sum \dfrac{Q_{i}}{T_{i}} < 0$ & Procesos irreversibles \\
%\vspace{.1cm}		Entropía & $dS =\dfrac{\delta Q}{T} \hspace{.4cm} T=cte$\\
%	\end{tabular}
%	\end{tcolorbox}	


%El minipage es para que quede el cuadro en toda la pagina, pero tengo que agregarle dirección horizontal negativa porque me lo hace en la segunda columna (columna de la derecha)
\hspace{-9.1cm}
\begin{minipage}{17cm}
	\begin{center}
		\unidad{5}{Segundo Principio}
		\vspace{.4cm}
	\end{center}
		\begin{tcolorbox}[colback=white!97!brown, colframe=brown!15!gray]
		\begin{center}
			\emph{La transferencia de calor y trabajo no es equivalente, existen pérdidas.}\\
			
			$Q_{1}=W+Q_{2}$
		\end{center}
		\begin{tabular}{r l r l }
			\vspace{.1cm}		Rendimiento & $\eta = \dfrac{\textup{energía útil}}{\textup{energía absorbida}}$ & 	\multicolumn{2}{c}{Rendimientos isentrópicos}\\
			\vspace{.1cm}		Ciclo de Carnot & $\eta = 1 - \dfrac{Q_{2}}{Q_{1}}=1-\dfrac{T_{2}}{T_{1}}$ & Turbina & $\eta_{s} =\dfrac{W_{real}}{W_{s}}=\dfrac{h_{1}-h_{2r}}{h_{1}-h_{2s}}$ \\
			\vspace{.1cm}		Teorema de Clausius & $\displaystyle\sum \dfrac{Q_{i}}{T_{i}} \leq 0$ &
			Compresor & $\eta_{s}=\dfrac{W_{s}}{W_{real}}=\dfrac{h_{2s}-h_{1}}{h_{2r}-h_{1}}$	\\
			\vspace{.1cm}		$\displaystyle\sum \dfrac{Q_{i}}{T_{i}} = 0$ & Procesos reversibles &&\\
			\vspace{.1cm}		$\displaystyle\sum \dfrac{Q_{i}}{T_{i}} < 0$ & Procesos irreversibles&& \\
			\vspace{.1cm}		Entropía & $dS =\dfrac{\delta Q}{T} \hspace{.4cm} T=cte$&&\\
		\end{tabular}
%				\begin{tabular}{r l}
%			\vspace{.1cm}		Rendimiento & $\eta = \dfrac{\textup{energía útil}}{\textup{energía absorbida}}$ \\
%			\vspace{.1cm}		Ciclo de Carnot & $\eta = 1 - \dfrac{Q_{2}}{Q_{1}}=1-\dfrac{T_{2}}{T_{1}}$ \\
%			\vspace{.1cm}		Teorema de Clausius & $\displaystyle\sum \dfrac{Q_{i}}{T_{i}} \leq 0$\\
%			\vspace{.1cm}		$\displaystyle\sum \dfrac{Q_{i}}{T_{i}} = 0$ & Procesos reversibles \\
%			\vspace{.1cm}		$\displaystyle\sum \dfrac{Q_{i}}{T_{i}} < 0$ & Procesos irreversibles \\
%			\vspace{.1cm}		Entropía & $dS =\dfrac{\delta Q}{T} \hspace{.4cm} T=cte$\\
%		\end{tabular}
\end{tcolorbox}
\end{minipage}	
%
%\begin{center}
%	Rendimientos isentrópicos
%\end{center}
%\begin{tabular}{r l}
%	\vspace{.1cm}		Turbina & $\eta = \dfrac{\textup{W real turbina}}{\textup{W isentrópico turbina}}$ \\
%	\vspace{.1cm}		& $\dfrac{W_{real}}{W_{s}}=\dfrac{h_{1}-h_{2r}}{h_{1}-h_{2s}}$ \\
%	\vspace{.1cm}		Compresor & $\eta = \dfrac{\textup{W isentrópico compresor}}{\textup{W real compresor}}=$ \\
%	\vspace{.1cm}		& $\dfrac{W_{s}}{W_{real}}=\dfrac{h_{2s}-h_{1}}{h_{2r}-h_{1}}$ \\
%	
%\end{tabular}
%\unidad{5}{Segundo principio}
%\begin{tcolorbox}[colback=white!97!brown, colframe=brown!15!gray]
%	\begin{center}
%		\emph{La transferencia de calor y trabajo no es equivalente, existen pérdidas.}\\
%		
%		$Q_{1}=W+Q_{2}$
%	\end{center}
%	\begin{tabular}{r l}
%		\vspace{.1cm}		Rendimiento & $\eta = \dfrac{\textup{energía útil}}{\textup{energía absorbida}}$ \\
%		\vspace{.1cm}		Ciclo de Carnot & $\eta = 1 - \dfrac{Q_{2}}{Q_{1}}=1-\dfrac{T_{2}}{T_{1}}$ \\
%		\vspace{.1cm}		Teorema de Clausius & $\displaystyle\sum \dfrac{Q_{i}}{T_{i}} \leq 0$\\
%		\vspace{.1cm}		$\displaystyle\sum \dfrac{Q_{i}}{T_{i}} = 0$ & Procesos reversibles \\
%		\vspace{.1cm}		$\displaystyle\sum \dfrac{Q_{i}}{T_{i}} < 0$ & Procesos irreversibles \\
%		\vspace{.1cm}		Entropía & $dS =\dfrac{\delta Q}{T} \hspace{.4cm} T=cte$\\
%	\end{tabular}
%\end{tcolorbox}	
\unidad{6}{Aire húmedo}
\begin{tcolorbox}
	\textbf{Psicrosometría}\\ 
	
	\begin{tabular}{l r}
		$P_{bar}=P_{aire}+P_{agua}$& 
		$ h=~h_{a}+\omega h_{v~sat}$\\
		$HR=\dfrac{m_{v}}{m_{sat}}=\dfrac{P_{v}}{P_{sat}}$&$\omega=\dfrac{m_{v}}{m_{a}}=\dfrac{0.622~p_{v}}{P~-~P_{v}}$
	\end{tabular}
	\begin{center}
			Ecuaciones de aire seco:
	\end{center}
	\begin{tabular}{l c}
		\textit{Masa de aire seco:}& $\sum_{ent}\dot{m}_{a}=\sum_{sal}\dot{m}_{a}$\\
		\textit{Masa de agua:} & $\sum_{ent}\dot{m}_{w}=\sum_{sal}\dot{m}_{w}$ \\
		&  $\sum_{ent}\dot{m}_{a} \omega=\sum_{sal}\dot{m}_{a}\omega$\\
		\textit{Energía:} &\\	
	\end{tabular}
	\begin{center}
		 $ \dot{Q}_{ent}+\dot{W}_{ent}+\sum_{ent} \dot{m}h= \dot{Q}_{sal}+\dot{W}_{sal}+\sum_{sal} \dot{m}h$
	\end{center}

\end{tcolorbox}
		
		
	
	\unidad{16?}{Transferencia calor}
	\vspace{0.1cm}
	\begin{tcolorbox}
	\textbf{Conducción de calor a través de paredes planas y compuestas.}
	\begin{center}
		$Q=\dfrac{A.\Delta T}{\dfrac{1}{h}+\dfrac{L}{k}+\dfrac{1}{h}}$
	\end{center}

	\textbf{Conducción de calor a través de cilindros huecos y compuestos.}
	\begin{center}
		$Q=\dfrac{2 \pi.L.\Delta T}{\dfrac{1}{h.r_{i}}+\dfrac{ln\left(\frac{r_{f}}{r_{i}}\right)}{k}+\dfrac{1}{h.r_{f}}}$
	\end{center}
	\end{tcolorbox}
\end{document}


