\thispagestyle{empty}
\vspace{2cm}
\begin{minipage}{.45\linewidth}
	\includegraphics[width = \linewidth]{{../../../utn}}
\end{minipage}\hfill
\begin{minipage}{.45\linewidth}	
	\begin{flushright}
		{\Large \textbf{Ingeniería Electromecánica}} \\
		\textbf{\curso} \\  Diseño Curricular: 2004 - Ord. N°1029 \\ Facultad Regional Reconquista
	\end{flushright}
	
\end{minipage}

\vspace{2cm}

\begin{minipage}{0.85\textwidth}
	{\fontsize{30}{30}\selectfont\textsc{\materia}\par\vspace{1em}\par}
\end{minipage}
\vspace{.5cm}

{\huge \textbf{Apunte para estudiantes}}

\vspace{2cm}

\begin{minipage}{0.85\textwidth}
	El presente documento fue elaborado por un grupo de estudiantes con el objetivo de crear un apunte completo y específico de la materia \materia.\\
	
	Este material abarca los temas más relevantes de la asignatura e incluye ejemplos ilustrativos y claros para facilitar su comprensión.\\
	
	Esperamos que este resumen sea de gran utilidad. Si tienen recomendaciones, sugerencias, correcciones o desean aportar algo, los invitamos a dejar su comentario. Y si les resulta útil, ¡compártanlo!
\end{minipage}

\vspace{3cm}

\begin{minipage}{0.7\textwidth}
	\textsl{\autor}
\end{minipage}

\newpage